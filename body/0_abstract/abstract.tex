\chapter{Abstract}
\label{ch:abstract}
The Standard Model (SM) of particle physics has been the dominant theory of all fundamental physics sans gravity for a good 50-or-so years.
The SM provides a unifying mathematical framework, explanations for various atomic and subatomic phenomena,
and a plethora of predictions, many of which have since been verified by experiments to high precision.
Nevertheless, there are still questions the SM still does not have an answer for, 
one of them being the imbalance of naturally existing matter and antimatter,
also known as the Baryon Asymmetry of the Universe (BAU).
Motivated by this question (among others), there have been various theoretical attempts at extending beyond the Standard Model (BSM).
One family of these extensions are the Two-Higgs Doublet Models (2HDMs), which propose an extra Higgs doublet with varying properties.
Among these 2HDMs, one type does not have any assumed symmetry or coupling constraints, and is known as the General Two Higgs Doublet Model ({\gthdm}).
A key feature of such theories attempting to address BAU is large charge-parity violation (CPV).
On the experimental front, both the lack of new collider observations as well as the smallness of low-energy precision measurements put heavy constraints on BSM CPV.
In particular, the electric dipole moments (EDMs) of various fundamental particles provide a \textit{litmus test} for CPV effects.
This study is an examination and analysis of various EDMs under the {\gthdm} framework. 
We start with the electron EDM ({\eedm}), introducing a \textit{cancellation mechanism} that allows for an adequate parameter space to address the BAU issue while evading the current EDM bounds.
We then extend the framework to the muon and the tau, and present both general and cancellation-imposed results for the EDM of heavier leptons.
We also probe the effects of the light quark EDMs through the neutron EDM ({\nedm}), taking into account gluon-related contributions.
At last, we present a combined scenario analysis of the {\eedm} and the {\nedm}, signifying that experimental verification, or even \textit{discovery}, is achieveable within the next decade or two.