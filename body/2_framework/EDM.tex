\chapter{Electric Dipole Moment}
\label{ch:EDM}

The effective interaction term that produces EDM \(d_{f} \) for a fermion \(f \) is the dimension-5 operator
\begin{equation}
  -\frac{i}{2}d_{f}\left(\bar{f}\sigma^{\mu\nu}\gamma_{5}f\right)F_{\mu\nu}.
\end{equation}

In {\gthdm}, the first finite contribution to EDM appears at one-loop.
The dipole operator is chirality violating, so an additional mass insertion is required on the fermion line to obtain the correct chiral structure.
This means that those one-loop diagrams with lighter leptons in the loop are chirally suppressed.
The next contribution to this operator is the two-loop Barr-Zee diagram. 
Naively, one would directly assume these to be loop-suppressed. 
However, the two-loop diagram having only one chirality flip, compared to three chirality flips for the one-loop diagram, 
effectively compensates for the loop suppression.

\begin{figure}[p]
	\centering
	\begin{fmffile}{Barr-Zee_General}
		\begin{fmfgraph*}(300, 200)
			\fmfleft{i1,i2}
	   		\fmfright{o1,o2}
	   		\fmf{plain}{i1,v1}
	   		\fmf{fermion,tension=0.25}{v1,v2}
	   		\fmf{plain}{v2,o1}
			\fmffreeze
			\fmfpoly{plain,smooth,filled=shaded,tension=0.8}{v3,v4,v5,v6}
			% \fmf{plain,right}{v3,v4}
			% \fmf{plain,right}{v4,v5}
			% \fmf{plain,right}{v5,v6}
			% \fmf{plain,right}{v6,v3}
			\fmf{scalar}{v1,v3}
			\fmf{boson}{v4,v2}
			\fmf{photon,tension=1.25}{v5,o2}
			\fmf{phantom,tension=1.25}{i2,v6}
			\fmffreeze
			% Add our labels
			% \fmflabel{$l$}{i1}
			% \fmflabel{$l$}{o1}
			% \fmflabel{$\gamma$}{o2}
			\fmfiv{label=$f$,label.angle=60}{vloc(__i1)}
			\fmfiv{label=$f$,label.angle=120}{vloc(__o1)}
			\fmfiv{label=$\gamma$,label.angle=0}{vloc(__o2)}
			\fmfi{phantom,label=$\phi$,label.side=left}{vpath(__v1,__v3)}
			\fmfi{phantom,label=$V$,label.side=left}{vpath(__v4,__v2)}
			% \fmfi{phantom,label=$f$,label.side=right,label.dist=0.01w}{vpath(__v3,__v4)}
		\end{fmfgraph*}
	\end{fmffile}

	\caption{Two-loop Barr-Zee diagram}
	\label{fig:BarrZee_general}
\end{figure}

It is straightforward yet tedious to calculate the two-loop Barr-Zee diagrams analytically, but it can be done nonetheless, 
as seen in the original paper by Barr and Zee~\cite{BarrZee} for neutral scalar contributions with a top quark or gauge boson in the loop,
as well as later extensions~\cite{MoreBarrZee} to other loop diagrams.
The final formulae for the various Barr-Zee diagrams are as follows, following the notations of~\cite{Abe},

\begin{align}\label{eq:BarrZee-phiG-toploop}
	(d^{\phi G}_{l})_{t} = -&\frac{e\,m_{l}}{(4\pi)^{4}}\sqrt{2}G_{F}\sum_{\phi=h,H,A}\sum_{G=\gamma,Z}N_{c}Q_{t}(g_{Gll}^{L}+g_{Gll}^{R}) \nonumber \\
	& \times \left[\frac{g_{\phi ll}^{A}}{m_{l}/v}\frac{g_{\phi tt}^{V}}{m_{t}/v}\mathcal{I}_{1}^{G}(m_{t}, m_{\phi}) + \frac{g_{\phi ll}^{V}}{m_{l}/v}\frac{g_{\phi tt}^{A}}{m_{t}/v}\mathcal{I}_{2}^{G}(m_{t}, m_{\phi})\right]
\end{align}
where
\begin{align}
	\mathcal{I}_{1}^{G}(m_{t},m_{\phi}) = (g_{Gtt}^{L}+g_{Gtt}^{A})\frac{m_{t}^{2}}{m_{\phi}^{2}-m_{G}^{2}}
	\left(-2\frac{m_{G}^{2}}{m_{t}^{2}} f\left(\frac{m_{t}^{2}}{m_{G}^{2}}\right) + 2\frac{m_{\phi}^{2}}{m_{t}^{2}} f\left(\frac{m_{t}^{2}}{m_{\phi}^{2}}\right)\right) \nonumber \\
	\mathcal{I}_{2}^{G}(m_{t},m_{\phi}) = (g_{Gtt}^{L}+g_{Gtt}^{A})\frac{m_{t}^{2}}{m_{\phi}^{2}-m_{G}^{2}}
	\left(-2\frac{m_{G}^{2}}{m_{t}^{2}} g\left(\frac{m_{t}^{2}}{m_{G}^{2}}\right) + 2\frac{m_{\phi}^{2}}{m_{t}^{2}} g\left(\frac{m_{t}^{2}}{m_{\phi}^{2}}\right)\right)
\end{align}

\begin{equation}\label{eq:BarrZee-phiG-Wloop}
	(d^{\phi G}_{l})_{W} = +\frac{e\,m_{l}}{(4\pi)^{4}}\sqrt{2}G_{F}\sum_{\phi=h,H,A}\sum_{G=\gamma,Z}(g_{Gll}^{L}+g_{Gll}^{R})\frac{g_{\phi ll}^{A}}{m_{l}/v}\frac{g_{WW\phi}}{2m_{W}^{2}/v}\mathcal{I}_{W}^{G}(m_{\phi})
\end{equation}
where
\begin{align}
	\mathcal{I}_{W}^{G}(m_{\phi}) = &g_{WWG}\frac{2m_{W}^{2}}{m_{\phi}^{2}-m_{G}^{2}} \nonumber \\
	& \times \left[-\frac{1}{4}\left\{\left(6-\frac{m_{G}^{2}}{m_{W}^{2}}\right) + \left(1-\frac{m_{G}^{2}}{2m_{W}^{2}}\right)\frac{m_{\phi}^{2}}{m_{W}^{2}}\right\}
	\left[-2\frac{m_{\phi}^{2}}{m_{W}^{2}} f\left(\frac{m_{W}^{2}}{m_{\phi}^{2}}\right) + 2\frac{m_{G}^{2}}{m_{W}^{2}} f\left(\frac{m_{W}^{2}}{m_{G}^{2}}\right)\right]\right. \nonumber \\
	&\left. \quad + \left\{\left(-4+\frac{m_{G}^{2}}{m_{W}^{2}}\right) + \frac{1}{4}\left(\left(6-\frac{m_{G}^{2}}{m_{W}^{2}}\right) + \left(1-\frac{m_{G}^{2}}{2m_{W}^{2}}\right)\frac{m_{\phi}^{2}}{m_{W}^{2}}\right)\right\}
	\left[-2\frac{m_{\phi}^{2}}{m_{W}^{2}} g\left(\frac{m_{W}^{2}}{m_{\phi}^{2}}\right) + 2\frac{m_{G}^{2}}{m_{W}^{2}} g\left(\frac{m_{W}^{2}}{m_{G}^{2}}\right)\right]\right]
\end{align}

\begin{equation}\label{eq:BarrZee-phiG-cHloop}
	(d^{\phi G}_{l})_{H^{\pm}} = +\frac{e\,m_{l}}{(4\pi)^{4}}\sqrt{2}G_{F}\sum_{\phi=h,H,A}\sum_{G=\gamma,Z}(g_{Gll}^{L}+g_{Gll}^{R})\frac{g_{\phi ll}^{A}}{m_{l}/v}\frac{g_{\phi H^{+}H^{-}}}{v}\mathcal{I}_{3}^{G}(m_{H^{\pm}}, m_{\phi})
\end{equation}
where
\begin{align}
	\mathcal{I}_{3}^{G}(m_{H^{\pm}}, m_{\phi}) =& -\frac{1}{2}g_{GH^{+}H^{-}}\frac{v^{2}}{m_{\phi}^{2}-m_{G}^{2}} \nonumber \\
	& \times \left[\left(-2\frac{m_{G}^{2}}{m_{H^{\pm}}^{2}} f\left(\frac{m_{H^{\pm}}^{2}}{m_{G}^{2}}\right) + 2\frac{m_{\phi}^{2}}{m_{H^{\pm}}^{2}} f\left(\frac{m_{H^{\pm}}^{2}}{m_{\phi}^{2}}\right)\right)
	-\left(-2\frac{m_{G}^{2}}{m_{H^{\pm}}^{2}} g\left(\frac{m_{H^{\pm}}^{2}}{m_{G}^{2}}\right) + 2\frac{m_{\phi}^{2}}{m_{H^{\pm}}^{2}} g\left(\frac{m_{H^{\pm}}^{2}}{m_{\phi}^{2}}\right)\right)\right]
\end{align}

\begin{equation}\label{eq:BarrZee-cHW-tbloop}
	(d^{H^{+}W^{+}}_{l})_{t/b} = 
\end{equation}

\begin{equation}\label{eq:BarrZee-cHW-Wloop}
	(d^{H^{+}W^{+}}_{l})_{W} = 
\end{equation}

\begin{equation}\label{eq:BarrZee-cHW-cHloop}
	(d^{H^{+}W^{+}}_{l})_{H^{+}} = 
\end{equation}

with loop functions

\begin{align}
	f(a) &= \frac{1}{2} a \int_{0}^{1}\dd{z}\frac{1-2z(1-z)}{z(1-z)-a}\log{\frac{z(1-z)}{a}} \nonumber \\
	g(a) &= \frac{1}{2} a \int_{0}^{1}\dd{z}\frac{1}{z(1-z)-a}\log{\frac{z(1-z)}{a}}
\end{align}

\begin{align}
	T(a)= \nonumber \\
	B(a)=
\end{align}

For quarks, they participate in the strong interaction, so there will be QCD-related effects.
This can be found in two additional terms in the Lagrangian: 
the chromo-EDM \(\tilde{d}_{f} \) for fermion \(f \), and the Weinberg term \(C_{W} \) for gluon interactions~\cite{Weinberg89}, written as
\begin{equation}
  -\frac{i g_{s}}{2}\tilde{d_{f}}\left(\bar{f}\sigma^{\mu\nu}T^{a}\gamma_{5}f\right)G^{a}_{\mu\nu}\quad \qquad \quad -\frac{1}{3}C_Wf^{abc}G^{a}_{\mu\sigma}G^{b,\sigma}_{\nu}\tilde{G}^{c,\mu\nu}
\end{equation}
The formulae for calculating the cEDM are

\begin{equation}
	\tilde{d_{f}} = +\frac{m_{f}}{(4\pi)^{4}}\sqrt{2}G_{F}\sum_{\phi=h,H,A}2g_{s}^{2}\frac{m_{f}^{2}}{m_{\phi}^{2}}
	\left[\frac{g_{\phi ff}^{A}}{m_{f}/v}\frac{g_{\phi tt}^{V}}{m_{t}/v}\left(-2\frac{m_{\phi}^{2}}{m_{t}^{2}} f\left(\frac{m_{t}^{2}}{m_{\phi}^{2}}\right)\right) 
	+ \frac{g_{\phi ff}^{V}}{m_{f}/v}\frac{g_{\phi tt}^{A}}{m_{t}/v}\left(-2\frac{m_{\phi}^{2}}{m_{t}^{2}} g\left(\frac{m_{t}^{2}}{m_{\phi}^{2}}\right)\right)\right]
\end{equation}

\begin{equation}
	C_{W} = 
\end{equation}

The contribution of the Weinberg diagram can be evaluated using QCD running.

\begin{align}
	\frac{d_{f}(\mu_{h})}{2} &= \\
	\frac{\tilde{d_{f}}(\mu_{h})}{2} &= \\
	C_{W}(\mu_{h}) &= 
\end{align}

where constants.

\begin{figure}[p]
	\centering
	\begin{subfigure}[t]{0.45\textwidth}
		\centering
		\begin{fmffile}{BarrZee-phiG-toploop}
			\begin{fmfgraph*}(180, 120)
				\fmfleft{i1,i2}
		   		\fmfright{o1,o2}
		   		\fmf{plain}{i1,v1}
		   		\fmf{fermion,tension=0.25}{v1,v2}
		   		\fmf{plain}{v2,o1}
				\fmffreeze
				\fmfpoly{phantom,smooth,tension=0.8}{v3,v4,v5,v6}
				% \fmf{plain,right}{v4,v5}
				% \fmf{plain,right}{v5,v6}
				% \fmf{plain,right}{v6,v3}
				\fmf{dashes}{v1,v3}
				\fmf{boson}{v4,v2}
				\fmf{photon,tension=1.25}{v5,o2}
				\fmf{phantom,tension=1.25}{i2,v6}
				\fmffreeze
				\fmf{plain,right=0.414}{v3,v4,v5,v6,v3}
				% \fmffreeze
				% Add our labels
				% \fmflabel{$f$}{i1}
				% \fmflabel{$f$}{o1}
				% \fmflabel{$\gamma$}{o2}
				\fmfiv{label=$f$,label.angle=60}{vloc(__i1)}
				\fmfiv{label=$f$,label.angle=120}{vloc(__o1)}
				\fmfiv{label=$\gamma$,label.angle=0}{vloc(__o2)}
				\fmfi{phantom,label=$\phi=h,,H,,A$,label.side=left}{vpath(__v1,__v3)}
				\fmfi{phantom,label=$G=\gamma,,Z$,label.side=left}{vpath(__v4,__v2)}
				\fmfi{phantom,label=$t$,label.side=left}{vpath(__v3,__v4)}
			\end{fmfgraph*}
		\end{fmffile}
		\caption{Neutral Higgs, top loop}
		\label{fig:BarrZee-phiG-toploop}
	\end{subfigure}
	\begin{subfigure}[t]{0.45\textwidth}
		\centering
		\begin{fmffile}{BarrZee-phiG-Wloop}
			\begin{fmfgraph*}(180, 120)
				\fmfleft{i1,i2}
		   		\fmfright{o1,o2}
		   		\fmf{plain}{i1,v1}
		   		\fmf{fermion,tension=0.25}{v1,v2}
		   		\fmf{plain}{v2,o1}
				\fmffreeze
				\fmfpoly{phantom,smooth,tension=0.8}{v3,v4,v5,v6}
				\fmf{dashes}{v1,v3}
				\fmf{boson}{v4,v2}
				\fmf{photon,tension=1.25}{v5,o2}
				\fmf{phantom,tension=1.25}{i2,v6}
				\fmffreeze
				\fmf{boson,right=0.414}{v3,v4,v5,v6,v3}
				% \fmffreeze
				% Add our labels
				% \fmflabel{$f$}{i1}
				% \fmflabel{$f$}{o1}
				% \fmflabel{$\gamma$}{o2}
				\fmfiv{label=$f$,label.angle=60}{vloc(__i1)}
				\fmfiv{label=$f$,label.angle=120}{vloc(__o1)}
				\fmfiv{label=$\gamma$,label.angle=0}{vloc(__o2)}
				\fmfi{phantom,label=$\phi=h,,H,,A$,label.side=left}{vpath(__v1,__v3)}
				\fmfi{phantom,label=$G=\gamma,,Z$,label.side=left}{vpath(__v4,__v2)}
				\fmfi{phantom,label=$W$,label.side=left}{vpath(__v3,__v4)}
			\end{fmfgraph*}
		\end{fmffile}
		\caption{Neutral Higgs, \(W \) loop}
		\label{fig:BarrZee-phiG-Wloop}
	\end{subfigure}
	\begin{subfigure}[t]{0.45\textwidth}
		\centering
		\begin{fmffile}{BarrZee-phiG-cHloop}
			\begin{fmfgraph*}(180, 120)
				\fmfleft{i1,i2}
		   		\fmfright{o1,o2}
		   		\fmf{plain}{i1,v1}
		   		\fmf{fermion,tension=0.25}{v1,v2}
		   		\fmf{plain}{v2,o1}
				\fmffreeze
				\fmfpoly{phantom,smooth,tension=0.8}{v3,v4,v5,v6}
				\fmf{dashes}{v1,v3}
				\fmf{boson}{v4,v2}
				\fmf{photon,tension=1.25}{v5,o2}
				\fmf{phantom,tension=1.25}{i2,v6}
				\fmffreeze
				\fmf{dashes,right=0.414}{v3,v4,v5,v6,v3}
				% \fmffreeze
				% Add our labels
				% \fmflabel{$f$}{i1}
				% \fmflabel{$f$}{o1}
				% \fmflabel{$\gamma$}{o2}
				\fmfiv{label=$f$,label.angle=60}{vloc(__i1)}
				\fmfiv{label=$f$,label.angle=120}{vloc(__o1)}
				\fmfiv{label=$\gamma$,label.angle=0}{vloc(__o2)}
				\fmfi{phantom,label=$\phi=h,,H,,A$,label.side=left}{vpath(__v1,__v3)}
				\fmfi{phantom,label=$G=\gamma,,Z$,label.side=left}{vpath(__v4,__v2)}
				\fmfi{phantom,label=$H^{\pm}$,label.side=left}{vpath(__v3,__v4)}
			\end{fmfgraph*}
		\end{fmffile}
		\caption{Neutral Higgs, charged Higgs loop}
		\label{fig:BarrZee-phiG-cHloop}
	\end{subfigure}
	\begin{subfigure}[t]{0.45\textwidth}
		\centering
		\begin{fmffile}{BarrZee-cHW-tbloop}
			\begin{fmfgraph*}(180, 120)
				\fmfleft{i1,i2}
		   		\fmfright{o1,o2}
		   		\fmf{plain}{i1,v1}
		   		\fmf{fermion,tension=0.25}{v1,v2}
		   		\fmf{plain}{v2,o1}
				\fmffreeze
				\fmfpoly{phantom,smooth,tension=0.8}{v3,v4,v5,v6}
				\fmf{dashes}{v1,v3}
				\fmf{boson}{v4,v2}
				\fmf{photon,tension=1.25}{v5,o2}
				\fmf{phantom,tension=1.25}{i2,v6}
				\fmffreeze
				\fmf{plain,right=0.414}{v3,v4,v5,v6,v3}
				% \fmffreeze
				% Add our labels
				% \fmflabel{$f$}{i1}
				% \fmflabel{$f$}{o1}
				% \fmflabel{$\gamma$}{o2}
				\fmfiv{label=$f^{\uparrow\downarrow}$,label.angle=60}{vloc(__i1)}
				\fmfiv{label=$f^{\uparrow\downarrow}$,label.angle=120}{vloc(__o1)}
				\fmfiv{label=$\gamma$,label.angle=0}{vloc(__o2)}
				\fmfi{phantom,label=$H^{\pm}$,label.side=left}{vpath(__v1,__v3)}
				\fmfi{phantom,label=$f^{\downarrow\uparrow}$,label.side=left}{vpath(__v1,__v2)}
				\fmfi{phantom,label=$W^{\pm}$,label.side=left}{vpath(__v4,__v2)}
				\fmfi{phantom,label=$t/b$,label.side=left}{vpath(__v3,__v4)}
			\end{fmfgraph*}
		\end{fmffile}
		\caption{Charged Higgs, top/bottom loop}
		\label{fig:BarrZee-cHW-tbloop}
	\end{subfigure}
	\begin{subfigure}[t]{0.45\textwidth}
		\centering
		\begin{fmffile}{BarrZee_cHW_Wloop}
			\begin{fmfgraph*}(180, 120)
				\fmfleft{i1,i2}
		   		\fmfright{o1,o2}
		   		\fmf{plain}{i1,v1}
		   		\fmf{fermion,tension=0.25}{v1,v2}
		   		\fmf{plain}{v2,o1}
				\fmffreeze
				\fmfpoly{phantom,smooth,tension=0.8}{v3,v4,v5,v6}
				\fmf{dashes}{v1,v3}
				\fmf{boson}{v4,v2}
				\fmf{photon,tension=1.25}{v5,o2}
				\fmf{phantom,tension=1.25}{i2,v6}
				\fmffreeze
				\fmf{boson,right=0.414}{v3,v4,v5,v6,v3}
				% \fmffreeze
				% Add our labels
				% \fmflabel{$f$}{i1}
				% \fmflabel{$f$}{o1}
				% \fmflabel{$\gamma$}{o2}
				\fmfiv{label=$f^{\uparrow\downarrow}$,label.angle=60}{vloc(__i1)}
				\fmfiv{label=$f^{\uparrow\downarrow}$,label.angle=120}{vloc(__o1)}
				\fmfiv{label=$\gamma$,label.angle=0}{vloc(__o2)}
				\fmfi{phantom,label=$H^{\pm}$,label.side=left}{vpath(__v1,__v3)}
				\fmfi{phantom,label=$f^{\downarrow\uparrow}$,label.side=left}{vpath(__v1,__v2)}
				\fmfi{phantom,label=$W^{\pm}$,label.side=left}{vpath(__v4,__v2)}
				\fmfi{phantom,label=$W$,label.side=left}{vpath(__v3,__v4)}
			\end{fmfgraph*}
		\end{fmffile}
		\caption{Charged Higgs, \(W \) loop}
		\label{fig:BarrZee_cHW_Wloop}
	\end{subfigure}
	\begin{subfigure}[t]{0.45\textwidth}
		\centering
		\begin{fmffile}{BarrZee-cHW-cHloop}
			\begin{fmfgraph*}(180, 120)
				\fmfleft{i1,i2}
		   		\fmfright{o1,o2}
		   		\fmf{plain}{i1,v1}
		   		\fmf{fermion,tension=0.25}{v1,v2}
		   		\fmf{plain}{v2,o1}
				\fmffreeze
				\fmfpoly{phantom,smooth,tension=0.8}{v3,v4,v5,v6}
				\fmf{dashes}{v1,v3}
				\fmf{boson}{v4,v2}
				\fmf{photon,tension=1.25}{v5,o2}
				\fmf{phantom,tension=1.25}{i2,v6}
				\fmffreeze
				\fmf{dashes,right=0.414}{v3,v4,v5,v6,v3}
				% \fmffreeze
				% Add our labels
				% \fmflabel{$f$}{i1}
				% \fmflabel{$f$}{o1}
				% \fmflabel{$\gamma$}{o2}
				\fmfiv{label=$f^{\uparrow\downarrow}$,label.angle=60}{vloc(__i1)}
				\fmfiv{label=$f^{\uparrow\downarrow}$,label.angle=120}{vloc(__o1)}
				\fmfiv{label=$\gamma$,label.angle=0}{vloc(__o2)}
				\fmfi{phantom,label=$H^{\pm}$,label.side=left}{vpath(__v1,__v3)}
				\fmfi{phantom,label=$f^{\downarrow\uparrow}$,label.side=left}{vpath(__v1,__v2)}
				\fmfi{phantom,label=$W^{\pm}$,label.side=left}{vpath(__v4,__v2)}
				\fmfi{phantom,label=$H^{\pm}$,label.side=left}{vpath(__v3,__v4)}
			\end{fmfgraph*}
		\end{fmffile}
		\caption{Charged Higgs, charged Higgs loop}
		\label{fig:BarrZee-cHW-cHloop}
	\end{subfigure}
	\caption{Specific Two-loop Barr-Zee diagrams}
	\label{fig:BarrZee-specific}
\end{figure}
\begin{figure}[p]
	\centering
	\begin{subfigure}[t]{0.45\textwidth}
		\centering
		\begin{fmffile}{BarrZee-chromo}
			\begin{fmfgraph*}(180, 120)
				\fmfleft{i1,i2}
		   		\fmfright{o1,o2}
		   		\fmf{plain}{i1,v1}
		   		\fmf{fermion,tension=0.25}{v1,v2}
		   		\fmf{plain}{v2,o1}
				\fmffreeze
				\fmfpoly{phantom,smooth,tension=0.8}{v3,v4,v5,v6}
				\fmf{dashes}{v1,v3}
				\fmf{gluon}{v4,v2}
				\fmf{gluon,tension=1.25}{v5,o2}
				\fmf{phantom,tension=1.25}{i2,v6}
				\fmffreeze
				\fmf{plain,right=0.414}{v3,v4,v5,v6,v3}
				% \fmffreeze
				% Add our labels
				% \fmflabel{$f$}{i1}
				% \fmflabel{$f$}{o1}
				% \fmflabel{$\gamma$}{o2}
				\fmfiv{label=$f$,label.angle=60}{vloc(__i1)}
				\fmfiv{label=$f$,label.angle=120}{vloc(__o1)}
				\fmfiv{label=$g$,label.angle=0}{vloc(__o2)}
				\fmfi{phantom,label=$\phi=h,,H,,A$,label.side=left}{vpath(__v1,__v3)}
				\fmfi{phantom,label=$g$,label.side=left}{vpath(__v4,__v2)}
				\fmfi{phantom,label=$t$,label.side=left}{vpath(__v3,__v4)}
			\end{fmfgraph*}
		\end{fmffile}
		\caption{Barr-Zee diagram, gluons}
		\label{fig:BarrZee-chromo}
	\end{subfigure}
	\begin{subfigure}[t]{0.45\textwidth}
		\begin{fmffile}{Weinberg}
			\begin{fmfgraph*}(180, 120)
				\fmfleft{i1}
				\fmfright{o1,o2}
				\fmf{gluon,tension=1.25}{i1,v1}
				\fmfpolyn{phantom,smooth,tension=0.8}{v}{8}
				\fmf{gluon}{v4,o1}
				\fmf{gluon}{v6,o2}
				\fmffreeze
				\fmf{dashes}{v3,v7}
				\fmfcyclen{plain,right=0.199}{v}{8}
				% Add our labels
				\fmfiv{label=$g$,label.angle=60}{vloc(__i1)}
				\fmfiv{label=$g$,label.angle=60}{vloc(__o1)}
				\fmfiv{label=$g$,label.angle=-60}{vloc(__o2)}
				\fmfi{phantom,label=$\phi/H^{\pm}$,label.side=right}{vpath(__v3,__v7)}
				\fmfiv{label=$t$,label.angle=0}{vloc(__v1)}
				\fmfiv{label=$t/b$,label.angle=0}{vloc(__v5)}
				% \fmfi{phantom,label=$t$,label.side=right}{vpath(__v8,__v2)}
				% \fmfi{phantom,label=$t(b)$,label.side=right}{vpath(__v4,__v6)}
			\end{fmfgraph*}
		\end{fmffile}
		\caption{Weinberg diagram}
		\label{fig:Weinberg}
	\end{subfigure}
	\caption{Diagrams relevant to chromo-EDM}
	\label{fig:chromo-edm-all}
\end{figure}