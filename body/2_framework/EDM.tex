\chapter{Electric Dipole Moment}
\label{ch:EDM}

The effective interaction term that produces EDM \(d_{f} \) for a fermion \(f \) is the dimension-5 operator
\begin{equation}
  -\frac{i}{2}d_{f}\left(\bar{f}\sigma^{\mu\nu}\gamma_{5}f\right)F_{\mu\nu}.
\end{equation}

In {\gthdm}, the first finite contribution to EDM appears at one-loop.
The dipole operator is chirality violating, so an additional mass insertion is required on the fermion line to obtain the correct chiral structure.
This means that those one-loop diagrams with lighter leptons in the loop are chirally suppressed.
The next contribution to this operator is the two-loop Barr-Zee diagram. 
Naively, one would directly assume these to be loop-suppressed. 
However, the two-loop diagram having only one chirality flip, compared to three chirality flips for the one-loop diagram, 
effectively compensates for the loop suppression.

It is straightforward yet tedious to calculate the two-loop Barr-Zee diagrams analytically, but it can be done nonetheless, 
as seen in the original paper by Barr and Zee~\cite{BarrZee} for neutral scalar contributions with a top quark or gauge boson in the loop,
as well as later extensions~\cite{MoreBarrZee} to other loop diagrams.
The final formulae for the various Barr-Zee diagrams are as follows, following the notations of~\cite{Abe},

\begin{equation}\label{eq:EDM-neutralScalar-toploop}
	(d^{\phi G}_{l})_{t} = 
\end{equation}

\begin{equation}\label{eq:EDM-neutralScalar-Wloop}
	(d^{\phi G}_{l})_{W} = 
\end{equation}

\begin{equation}\label{eq:EDM-neutralScalar-cHloop}
	(d^{\phi G}_{l})_{H^{+}} = 
\end{equation}

\begin{equation}\label{eq:EDM-chargedScalar-tbloop}
	(d^{H^{+}W^{+}}_{l})_{t/b} = 
\end{equation}

\begin{equation}\label{eq:EDM-chargedScalar-Wloop}
	(d^{H^{+}W^{+}}_{l})_{W} = 
\end{equation}

\begin{equation}\label{eq:EDM-chargedScalar-cHloop}
	(d^{H^{+}W^{+}}_{l})_{H^{+}} = 
\end{equation}

with loop functions

\begin{equation}
	f(a)= , g(a)=
\end{equation}

\begin{equation}
	T(a)= , B(a)=
\end{equation}

For quarks, they participate in the strong interaction, so there will be QCD-related effects.
This can be found in two additional terms in the Lagrangian: 
the chromo-EDM \(\tilde{d}_{f} \) for fermion \(f \), and the Weinberg term \(C_{W} \) for gluon interactions~\cite{Weinberg89}, written as
\begin{equation}
  -\frac{i g_{s}}{2}\tilde{d_{f}}\left(\bar{f}\sigma^{\mu\nu}T^{a}\gamma_{5}f\right)G^{a}_{\mu\nu}\quad \qquad \quad -\frac{1}{3}C_Wf^{abc}G^{a}_{\mu\sigma}G^{b,\sigma}_{\nu}\tilde{G}^{c,\mu\nu}
\end{equation}
The formulae for calculating the cEDM are

\begin{equation}
	\tilde{d_{f}} = 
\end{equation}

\begin{equation}
	C_{W} = 
\end{equation}

The contribution of the Weinberg diagram can be evaluated using QCD running.

\begin{align}
	\frac{d_{f}(\mu_{h})}{2} &= \\
	\frac{\tilde{d_{f}}(\mu_{h})}{2} &= \\
	C_{W}(\mu_{h}) &= 
\end{align}

where constants.