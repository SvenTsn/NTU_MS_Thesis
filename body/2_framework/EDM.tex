\chapter{Electric Dipole Moment}
\label{ch:EDM}

\section{The electric dipole moment in particle physics}
Classically, the electric dipole moment of a system is found by 
\begin{equation}
	\vb{d} = \int (\vb{r}-\vb{r_{0}})\rho(\vb{r})\dd[3]{\vb{r}}
\end{equation}
with \(\vb{r_{0}} \) (check Griffiths/Jackson Multipole expansion).
Physically, it is inherently a property of composite systems.
For a system of zero net charge, it is related to the seperation of charges or nonuniformity in the charge distribution;
for a system of nonzero net charge, it is also related to the difference of the position of the point of reference with respect to the charge in question.
However, in particle physics, we are not dealing with composite systems, but the \textit{intrinsic} properties of the fundamental particles.
In the case of \textit{intrinsic} EDM, particle physicists are interested in the EDMs of various fermions.
Generally speaking, there are two ways to introduce this ``EDM term'', approaching the issue from slightly different angles.
\subsection{The Dirac method}
The first method is that of Dirac~\cite{Dirac1928DiracEquation}. 
In his 1928 paper, which gave birth to the famous Dirac equation, he also wrote down the equation for the electron in an arbitrary electromagnetic field.
Adapted into more modern notation, it is written as
\begin{equation}\label{eq:DiracEqwithEMField}
	\left[-(E-e\phi) + \vb{\alpha}\vdot(\vb{p}-e\vb{A}) + \beta m\right]\psi = 0
\end{equation}
where
\begin{equation}
	\vb{\alpha} = \left(\right), \quad \quad \beta = 
\end{equation}
in the Pauli-Dirac representation.
By ``squaring'' \eqnref{DiracEqwithEMField} one obtains
\begin{align}
	\left[\right]\left[\right]\psi &= 0 \nonumber \\
	\Longrightarrow \left[\right]\psi &= 0
\end{align}
After some matrix algebra, in particular \((\vb{\sigma}\vdot\vb{a})(\vb{\sigma}\vdot\vb{b}) = (\vb{a}\vdot\vb{b})I + i\vb{\sigma}\vdot(\vb{\vb{a}\cross\vb{b}}) \), we arrive at
\begin{equation}
	\left[-(E-e\phi)^{2} + (\vb{p}-e\vb{A})^{2} + m^{2} - e\vb{\sigma}\vdot\vb{B} - ie\gamma^{5}\vb{\sigma}\vdot\vb{E}\right]\psi = 0
\end{equation}
which introduces two additional terms, corresponding to the magnetic and the electric dipole moment of the electron, respectively.
At the time, Dirac thought the EDM term was merely a consequence of the ``squaring'' used in the derivation
arbitrarily introducing an imaginary term from a real Hamiltonian, and disregarded it.
It was only until 1958 when Salpeter~\cite{Salpeter1958EDMTerm} reintroduced this term in an ``interaction Lagrangian'' point of view.
\subsection{Form factor decomposition}
The second method is based on form factor decomposition. This method was recently seen in~\cite{Nowakowski2005FormFactor}.
We start by expressing the expectation value of the electromagnetic 4-current as 
\begin{equation}
	\mel{p_{f}}{j^{\mu}}{p_{i}} = \bar{u}(\vb{p_{f}})\mathcal{O}^{\mu}(l,q)u(\vb{p_{i}})
\end{equation}
where \(l \equiv p_{f} + p_{i} \) is the total 4-momentum, \(q \equiv p_{f} - p_{i}\) is the momentum transfer,
and \(\mathcal{O}^{\mu}(l,q) \) is an operator whose matrix element between the spinors is a Lorentz vector.
We then want to decompose \(\mathcal{O}^{\mu}(l,q) \) in terms of the Clifford algebra of the Dirac gamma matrices.
After identifying all possible combinations and contractions of \(\left\{q^{\mu}, l^{\mu}, \gamma^{\mu}, \gamma^{5}, \sigma^{\mu\nu}, \epsilon^{\mu\nu\alpha\beta}\right\} \),
applying various Gordon-like identities, and imposing gauge invariance\footnote{Note that without gauge invariance imposed, there ought to be six independent terms, which is the case for the Weak current.} \(q_{\mu}j^{\mu} = 0 \), we arrive at four independent terms
\begin{equation}
	\bar{u}(\vb{p_{f}})\mathcal{O}^{\mu}(l,q)u(\vb{p_{i}})
	= \bar{u}(\vb{p_{f}})\left\{\right\}u(\vb{p_{i}})
\end{equation}

We then couple this with the electromagnetic potential \(A_{\mu} \) to obtain some physical insight on this result.
Taking the non-relativistic limit \((q^{2} = 0) \), we can identify 
\begin{equation}
	F_{1}(0) = Q, \quad \frac{1}{2m}\left(F_{1}(0)+F_{2}(2)\right) = \mu, \quad -\frac{1}{2m}F_{3}(0) = d
\end{equation}
are the charge, magnetic dipole moment, and electric dipole moment, respectively.
Transforming the coupled current plus EM field back into position space, we can obtain the addition to the interaction Lagranginan for each term as well.

Regardless of approach, the end result, in terms of quantum field theory, is the inclusion of an effective dimension-5 interaction operator to the EM Lagrangian,
\begin{equation}
  -\frac{i}{2}d_{f}\left(\bar{f}\sigma^{\mu\nu}\gamma_{5}f\right)F_{\mu\nu}.
\end{equation}
that produces EDM \(d_{f} \) for a fermion \(f \), where \(F_{\mu\nu} \) is the electromagnetic field strength tensor.

\section{G2HDM contributions to EDMs}

In {\gthdm}, the first finite contribution to EDM appears at one-loop.
The dipole operator is chirality violating, so an additional mass insertion is required on the fermion line to obtain the correct chiral structure.
This means that those one-loop diagrams with lighter leptons in the loop are chirally suppressed.
The next contribution to this operator is the two-loop Barr-Zee diagram. 
Naively, one would directly assume these to be loop-suppressed. 
However, the two-loop diagram having only one chirality flip, compared to three chirality flips for the one-loop diagram, 
effectively compensates for the loop suppression.

\begin{figure}[p]
	\centering
	\begin{fmffile}{Barr-Zee_General}
		\begin{fmfgraph*}(300, 200)
			\fmfleft{i1,i2}
	   		\fmfright{o1,o2}
	   		\fmf{plain}{i1,v1}
	   		\fmf{fermion,tension=0.25}{v1,v2}
	   		\fmf{plain}{v2,o1}
			\fmffreeze
			\fmfpoly{plain,smooth,filled=shaded,tension=0.8}{v3,v4,v5,v6}
			% \fmf{plain,right}{v3,v4}
			% \fmf{plain,right}{v4,v5}
			% \fmf{plain,right}{v5,v6}
			% \fmf{plain,right}{v6,v3}
			\fmf{scalar}{v1,v3}
			\fmf{boson}{v4,v2}
			\fmf{photon,tension=1.25}{v5,o2}
			\fmf{phantom,tension=1.25}{i2,v6}
			\fmffreeze
			% Add our labels
			% \fmflabel{$l$}{i1}
			% \fmflabel{$l$}{o1}
			% \fmflabel{$\gamma$}{o2}
			\fmfiv{label=$f$,label.angle=60}{vloc(__i1)}
			\fmfiv{label=$f$,label.angle=120}{vloc(__o1)}
			\fmfiv{label=$\gamma$,label.angle=0}{vloc(__o2)}
			\fmfi{phantom,label=$\phi$,label.side=left}{vpath(__v1,__v3)}
			\fmfi{phantom,label=$V$,label.side=left}{vpath(__v4,__v2)}
			% \fmfi{phantom,label=$f$,label.side=right,label.dist=0.01w}{vpath(__v3,__v4)}
		\end{fmfgraph*}
	\end{fmffile}

	\caption{Two-loop Barr-Zee diagram}
	\label{fig:BarrZee_general}
\end{figure}

It is straightforward yet tedious to calculate the two-loop Barr-Zee diagrams analytically, but it can be done nonetheless, 
as seen in the original paper by Barr and Zee~\cite{BarrZee} for neutral scalar contributions with a top quark or gauge boson in the loop,
as well as later extensions~\cite{MoreBarrZee} to other loop diagrams.
The final formulae for the various Barr-Zee diagrams are as follows, following the notations of~\cite{Abe},

\begin{align}\label{eq:BarrZee-phiG-toploop}
	(d^{\phi G}_{l})_{t} = -&\frac{e\,m_{l}}{(4\pi)^{4}}\sqrt{2}G_{F}\sum_{\phi=h,H,A}\sum_{G=\gamma,Z}N_{c}Q_{t}(g_{Gll}^{L}+g_{Gll}^{R}) \nonumber \\
	& \times \left[\frac{g_{\phi ll}^{A}}{m_{l}/v}\frac{g_{\phi tt}^{V}}{m_{t}/v}\mathcal{I}_{1}^{G}(m_{t}, m_{\phi}) + \frac{g_{\phi ll}^{V}}{m_{l}/v}\frac{g_{\phi tt}^{A}}{m_{t}/v}\mathcal{I}_{2}^{G}(m_{t}, m_{\phi})\right]
\end{align}
where
\begin{align}
	\mathcal{I}_{1}^{G}(m_{t},m_{\phi}) = (g_{Gtt}^{L}+g_{Gtt}^{A})\frac{m_{t}^{2}}{m_{\phi}^{2}-m_{G}^{2}}
	\left(-2\frac{m_{G}^{2}}{m_{t}^{2}} f\left(\frac{m_{t}^{2}}{m_{G}^{2}}\right) + 2\frac{m_{\phi}^{2}}{m_{t}^{2}} f\left(\frac{m_{t}^{2}}{m_{\phi}^{2}}\right)\right) \nonumber \\
	\mathcal{I}_{2}^{G}(m_{t},m_{\phi}) = (g_{Gtt}^{L}+g_{Gtt}^{A})\frac{m_{t}^{2}}{m_{\phi}^{2}-m_{G}^{2}}
	\left(-2\frac{m_{G}^{2}}{m_{t}^{2}} g\left(\frac{m_{t}^{2}}{m_{G}^{2}}\right) + 2\frac{m_{\phi}^{2}}{m_{t}^{2}} g\left(\frac{m_{t}^{2}}{m_{\phi}^{2}}\right)\right)
\end{align}

\begin{equation}\label{eq:BarrZee-phiG-Wloop}
	(d^{\phi G}_{l})_{W} = +\frac{e\,m_{l}}{(4\pi)^{4}}\sqrt{2}G_{F}\sum_{\phi=h,H,A}\sum_{G=\gamma,Z}(g_{Gll}^{L}+g_{Gll}^{R})\frac{g_{\phi ll}^{A}}{m_{l}/v}\frac{g_{WW\phi}}{2m_{W}^{2}/v}\mathcal{I}_{W}^{G}(m_{\phi})
\end{equation}
where
\begin{align}
	\mathcal{I}_{W}^{G}(m_{\phi}) = &g_{WWG}\frac{2m_{W}^{2}}{m_{\phi}^{2}-m_{G}^{2}} \nonumber \\
	& \times \left[-\frac{1}{4}\left\{\left(6-\frac{m_{G}^{2}}{m_{W}^{2}}\right) + \left(1-\frac{m_{G}^{2}}{2m_{W}^{2}}\right)\frac{m_{\phi}^{2}}{m_{W}^{2}}\right\}
	\left[-2\frac{m_{\phi}^{2}}{m_{W}^{2}} f\left(\frac{m_{W}^{2}}{m_{\phi}^{2}}\right) + 2\frac{m_{G}^{2}}{m_{W}^{2}} f\left(\frac{m_{W}^{2}}{m_{G}^{2}}\right)\right]\right. \nonumber \\
	&\left. \quad + \left\{\left(-4+\frac{m_{G}^{2}}{m_{W}^{2}}\right) + \frac{1}{4}\left(\left(6-\frac{m_{G}^{2}}{m_{W}^{2}}\right) + \left(1-\frac{m_{G}^{2}}{2m_{W}^{2}}\right)\frac{m_{\phi}^{2}}{m_{W}^{2}}\right)\right\}
	\left[-2\frac{m_{\phi}^{2}}{m_{W}^{2}} g\left(\frac{m_{W}^{2}}{m_{\phi}^{2}}\right) + 2\frac{m_{G}^{2}}{m_{W}^{2}} g\left(\frac{m_{W}^{2}}{m_{G}^{2}}\right)\right]\right]
\end{align}

\begin{equation}\label{eq:BarrZee-phiG-cHloop}
	(d^{\phi G}_{l})_{H^{\pm}} = +\frac{e\,m_{l}}{(4\pi)^{4}}\sqrt{2}G_{F}\sum_{\phi=h,H,A}\sum_{G=\gamma,Z}(g_{Gll}^{L}+g_{Gll}^{R})\frac{g_{\phi ll}^{A}}{m_{l}/v}\frac{g_{\phi H^{+}H^{-}}}{v}\mathcal{I}_{3}^{G}(m_{H^{\pm}}, m_{\phi})
\end{equation}
where
\begin{align}
	\mathcal{I}_{3}^{G}(m_{H^{\pm}}, m_{\phi}) =& -\frac{1}{2}g_{GH^{+}H^{-}}\frac{v^{2}}{m_{\phi}^{2}-m_{G}^{2}} \nonumber \\
	& \times \left[\left(-2\frac{m_{G}^{2}}{m_{H^{\pm}}^{2}} f\left(\frac{m_{H^{\pm}}^{2}}{m_{G}^{2}}\right) + 2\frac{m_{\phi}^{2}}{m_{H^{\pm}}^{2}} f\left(\frac{m_{H^{\pm}}^{2}}{m_{\phi}^{2}}\right)\right)
	-\left(-2\frac{m_{G}^{2}}{m_{H^{\pm}}^{2}} g\left(\frac{m_{H^{\pm}}^{2}}{m_{G}^{2}}\right) + 2\frac{m_{\phi}^{2}}{m_{H^{\pm}}^{2}} g\left(\frac{m_{H^{\pm}}^{2}}{m_{\phi}^{2}}\right)\right)\right]
\end{align}

\begin{equation}\label{eq:BarrZee-cHW-tbloop}
	(d^{H^{+}W^{+}}_{l})_{t/b} = 
\end{equation}

\begin{equation}\label{eq:BarrZee-cHW-Wloop}
	(d^{H^{+}W^{+}}_{l})_{W} = 
\end{equation}

\begin{equation}\label{eq:BarrZee-cHW-cHloop}
	(d^{H^{+}W^{+}}_{l})_{H^{+}} = 
\end{equation}

with loop functions

\begin{align}
	f(a) &= \frac{1}{2} a \int_{0}^{1}\dd{z}\frac{1-2z(1-z)}{z(1-z)-a}\log{\frac{z(1-z)}{a}} \nonumber \\
	g(a) &= \frac{1}{2} a \int_{0}^{1}\dd{z}\frac{1}{z(1-z)-a}\log{\frac{z(1-z)}{a}}
\end{align}

\begin{align}
	T(a)= \nonumber \\
	B(a)=
\end{align}

\section{Chromoelectric dipole moment}
For quarks, they participate in the strong interaction, so there will be QCD-related effects.
This can be found in two additional terms in the Lagrangian: 
the chromo-EDM \(\tilde{d}_{f} \) for fermion \(f \), and the Weinberg term \(C_{W} \) for gluon interactions~\cite{Weinberg89}, written as
\begin{equation}
  -\frac{i g_{s}}{2}\tilde{d_{f}}\left(\bar{f}\sigma^{\mu\nu}T^{a}\gamma_{5}f\right)G^{a}_{\mu\nu}\quad \qquad \quad -\frac{1}{3}C_Wf^{abc}G^{a}_{\mu\sigma}G^{b,\sigma}_{\nu}\tilde{G}^{c,\mu\nu}
\end{equation}
The formulae for calculating the cEDM are

\begin{equation}
	\tilde{d_{f}} = +\frac{m_{f}}{(4\pi)^{4}}\sqrt{2}G_{F}\sum_{\phi=h,H,A}2g_{s}^{2}\frac{m_{f}^{2}}{m_{\phi}^{2}}
	\left[\frac{g_{\phi ff}^{A}}{m_{f}/v}\frac{g_{\phi tt}^{V}}{m_{t}/v}\left(-2\frac{m_{\phi}^{2}}{m_{t}^{2}} f\left(\frac{m_{t}^{2}}{m_{\phi}^{2}}\right)\right) 
	+ \frac{g_{\phi ff}^{V}}{m_{f}/v}\frac{g_{\phi tt}^{A}}{m_{t}/v}\left(-2\frac{m_{\phi}^{2}}{m_{t}^{2}} g\left(\frac{m_{t}^{2}}{m_{\phi}^{2}}\right)\right)\right]
\end{equation}

\begin{equation}
	C_{W} = 
\end{equation}

The contribution of the Weinberg diagram can be evaluated using QCD running.

\begin{align}
	\frac{d_{f}(\mu_{h})}{2} &= \\
	\frac{\tilde{d_{f}}(\mu_{h})}{2} &= \\
	C_{W}(\mu_{h}) &= 
\end{align}

where constants.

\begin{figure}[p]
	\centering
	\begin{subfigure}[t]{0.45\textwidth}
		\centering
		\begin{fmffile}{BarrZee-phiG-toploop}
			\begin{fmfgraph*}(180, 120)
				\fmfleft{i1,i2}
		   		\fmfright{o1,o2}
		   		\fmf{plain}{i1,v1}
		   		\fmf{fermion,tension=0.25}{v1,v2}
		   		\fmf{plain}{v2,o1}
				\fmffreeze
				\fmfpoly{phantom,smooth,tension=0.8}{v3,v4,v5,v6}
				% \fmf{plain,right}{v4,v5}
				% \fmf{plain,right}{v5,v6}
				% \fmf{plain,right}{v6,v3}
				\fmf{dashes}{v1,v3}
				\fmf{boson}{v4,v2}
				\fmf{photon,tension=1.25}{v5,o2}
				\fmf{phantom,tension=1.25}{i2,v6}
				\fmffreeze
				\fmf{plain,right=0.414}{v3,v4,v5,v6,v3}
				% \fmffreeze
				% Add our labels
				% \fmflabel{$f$}{i1}
				% \fmflabel{$f$}{o1}
				% \fmflabel{$\gamma$}{o2}
				\fmfiv{label=$f$,label.angle=60}{vloc(__i1)}
				\fmfiv{label=$f$,label.angle=120}{vloc(__o1)}
				\fmfiv{label=$\gamma$,label.angle=0}{vloc(__o2)}
				\fmfi{phantom,label=$\phi=h,,H,,A$,label.side=left}{vpath(__v1,__v3)}
				\fmfi{phantom,label=$G=\gamma,,Z$,label.side=left}{vpath(__v4,__v2)}
				\fmfi{phantom,label=$t$,label.side=left}{vpath(__v3,__v4)}
			\end{fmfgraph*}
		\end{fmffile}
		\caption{Neutral Higgs, top loop}
		\label{fig:BarrZee-phiG-toploop}
	\end{subfigure}
	\begin{subfigure}[t]{0.45\textwidth}
		\centering
		\begin{fmffile}{BarrZee-phiG-Wloop}
			\begin{fmfgraph*}(180, 120)
				\fmfleft{i1,i2}
		   		\fmfright{o1,o2}
		   		\fmf{plain}{i1,v1}
		   		\fmf{fermion,tension=0.25}{v1,v2}
		   		\fmf{plain}{v2,o1}
				\fmffreeze
				\fmfpoly{phantom,smooth,tension=0.8}{v3,v4,v5,v6}
				\fmf{dashes}{v1,v3}
				\fmf{boson}{v4,v2}
				\fmf{photon,tension=1.25}{v5,o2}
				\fmf{phantom,tension=1.25}{i2,v6}
				\fmffreeze
				\fmf{boson,right=0.414}{v3,v4,v5,v6,v3}
				% \fmffreeze
				% Add our labels
				% \fmflabel{$f$}{i1}
				% \fmflabel{$f$}{o1}
				% \fmflabel{$\gamma$}{o2}
				\fmfiv{label=$f$,label.angle=60}{vloc(__i1)}
				\fmfiv{label=$f$,label.angle=120}{vloc(__o1)}
				\fmfiv{label=$\gamma$,label.angle=0}{vloc(__o2)}
				\fmfi{phantom,label=$\phi=h,,H,,A$,label.side=left}{vpath(__v1,__v3)}
				\fmfi{phantom,label=$G=\gamma,,Z$,label.side=left}{vpath(__v4,__v2)}
				\fmfi{phantom,label=$W$,label.side=left}{vpath(__v3,__v4)}
			\end{fmfgraph*}
		\end{fmffile}
		\caption{Neutral Higgs, \(W \) loop}
		\label{fig:BarrZee-phiG-Wloop}
	\end{subfigure}
	\begin{subfigure}[t]{0.45\textwidth}
		\centering
		\begin{fmffile}{BarrZee-phiG-cHloop}
			\begin{fmfgraph*}(180, 120)
				\fmfleft{i1,i2}
		   		\fmfright{o1,o2}
		   		\fmf{plain}{i1,v1}
		   		\fmf{fermion,tension=0.25}{v1,v2}
		   		\fmf{plain}{v2,o1}
				\fmffreeze
				\fmfpoly{phantom,smooth,tension=0.8}{v3,v4,v5,v6}
				\fmf{dashes}{v1,v3}
				\fmf{boson}{v4,v2}
				\fmf{photon,tension=1.25}{v5,o2}
				\fmf{phantom,tension=1.25}{i2,v6}
				\fmffreeze
				\fmf{dashes,right=0.414}{v3,v4,v5,v6,v3}
				% \fmffreeze
				% Add our labels
				% \fmflabel{$f$}{i1}
				% \fmflabel{$f$}{o1}
				% \fmflabel{$\gamma$}{o2}
				\fmfiv{label=$f$,label.angle=60}{vloc(__i1)}
				\fmfiv{label=$f$,label.angle=120}{vloc(__o1)}
				\fmfiv{label=$\gamma$,label.angle=0}{vloc(__o2)}
				\fmfi{phantom,label=$\phi=h,,H,,A$,label.side=left}{vpath(__v1,__v3)}
				\fmfi{phantom,label=$G=\gamma,,Z$,label.side=left}{vpath(__v4,__v2)}
				\fmfi{phantom,label=$H^{\pm}$,label.side=left}{vpath(__v3,__v4)}
			\end{fmfgraph*}
		\end{fmffile}
		\caption{Neutral Higgs, charged Higgs loop}
		\label{fig:BarrZee-phiG-cHloop}
	\end{subfigure}
	\begin{subfigure}[t]{0.45\textwidth}
		\centering
		\begin{fmffile}{BarrZee-cHW-tbloop}
			\begin{fmfgraph*}(180, 120)
				\fmfleft{i1,i2}
		   		\fmfright{o1,o2}
		   		\fmf{plain}{i1,v1}
		   		\fmf{fermion,tension=0.25}{v1,v2}
		   		\fmf{plain}{v2,o1}
				\fmffreeze
				\fmfpoly{phantom,smooth,tension=0.8}{v3,v4,v5,v6}
				\fmf{dashes}{v1,v3}
				\fmf{boson}{v4,v2}
				\fmf{photon,tension=1.25}{v5,o2}
				\fmf{phantom,tension=1.25}{i2,v6}
				\fmffreeze
				\fmf{plain,right=0.414}{v3,v4,v5,v6,v3}
				% \fmffreeze
				% Add our labels
				% \fmflabel{$f$}{i1}
				% \fmflabel{$f$}{o1}
				% \fmflabel{$\gamma$}{o2}
				\fmfiv{label=$f^{\uparrow\downarrow}$,label.angle=60}{vloc(__i1)}
				\fmfiv{label=$f^{\uparrow\downarrow}$,label.angle=120}{vloc(__o1)}
				\fmfiv{label=$\gamma$,label.angle=0}{vloc(__o2)}
				\fmfi{phantom,label=$H^{\pm}$,label.side=left}{vpath(__v1,__v3)}
				\fmfi{phantom,label=$f^{\downarrow\uparrow}$,label.side=left}{vpath(__v1,__v2)}
				\fmfi{phantom,label=$W^{\pm}$,label.side=left}{vpath(__v4,__v2)}
				\fmfi{phantom,label=$t/b$,label.side=left}{vpath(__v3,__v4)}
			\end{fmfgraph*}
		\end{fmffile}
		\caption{Charged Higgs, top/bottom loop}
		\label{fig:BarrZee-cHW-tbloop}
	\end{subfigure}
	\begin{subfigure}[t]{0.45\textwidth}
		\centering
		\begin{fmffile}{BarrZee_cHW_Wloop}
			\begin{fmfgraph*}(180, 120)
				\fmfleft{i1,i2}
		   		\fmfright{o1,o2}
		   		\fmf{plain}{i1,v1}
		   		\fmf{fermion,tension=0.25}{v1,v2}
		   		\fmf{plain}{v2,o1}
				\fmffreeze
				\fmfpoly{phantom,smooth,tension=0.8}{v3,v4,v5,v6}
				\fmf{dashes}{v1,v3}
				\fmf{boson}{v4,v2}
				\fmf{photon,tension=1.25}{v5,o2}
				\fmf{phantom,tension=1.25}{i2,v6}
				\fmffreeze
				\fmf{boson,right=0.414}{v3,v4,v5,v6,v3}
				% \fmffreeze
				% Add our labels
				% \fmflabel{$f$}{i1}
				% \fmflabel{$f$}{o1}
				% \fmflabel{$\gamma$}{o2}
				\fmfiv{label=$f^{\uparrow\downarrow}$,label.angle=60}{vloc(__i1)}
				\fmfiv{label=$f^{\uparrow\downarrow}$,label.angle=120}{vloc(__o1)}
				\fmfiv{label=$\gamma$,label.angle=0}{vloc(__o2)}
				\fmfi{phantom,label=$H^{\pm}$,label.side=left}{vpath(__v1,__v3)}
				\fmfi{phantom,label=$f^{\downarrow\uparrow}$,label.side=left}{vpath(__v1,__v2)}
				\fmfi{phantom,label=$W^{\pm}$,label.side=left}{vpath(__v4,__v2)}
				\fmfi{phantom,label=$W$,label.side=left}{vpath(__v3,__v4)}
			\end{fmfgraph*}
		\end{fmffile}
		\caption{Charged Higgs, \(W \) loop}
		\label{fig:BarrZee_cHW_Wloop}
	\end{subfigure}
	\begin{subfigure}[t]{0.45\textwidth}
		\centering
		\begin{fmffile}{BarrZee-cHW-cHloop}
			\begin{fmfgraph*}(180, 120)
				\fmfleft{i1,i2}
		   		\fmfright{o1,o2}
		   		\fmf{plain}{i1,v1}
		   		\fmf{fermion,tension=0.25}{v1,v2}
		   		\fmf{plain}{v2,o1}
				\fmffreeze
				\fmfpoly{phantom,smooth,tension=0.8}{v3,v4,v5,v6}
				\fmf{dashes}{v1,v3}
				\fmf{boson}{v4,v2}
				\fmf{photon,tension=1.25}{v5,o2}
				\fmf{phantom,tension=1.25}{i2,v6}
				\fmffreeze
				\fmf{dashes,right=0.414}{v3,v4,v5,v6,v3}
				% \fmffreeze
				% Add our labels
				% \fmflabel{$f$}{i1}
				% \fmflabel{$f$}{o1}
				% \fmflabel{$\gamma$}{o2}
				\fmfiv{label=$f^{\uparrow\downarrow}$,label.angle=60}{vloc(__i1)}
				\fmfiv{label=$f^{\uparrow\downarrow}$,label.angle=120}{vloc(__o1)}
				\fmfiv{label=$\gamma$,label.angle=0}{vloc(__o2)}
				\fmfi{phantom,label=$H^{\pm}$,label.side=left}{vpath(__v1,__v3)}
				\fmfi{phantom,label=$f^{\downarrow\uparrow}$,label.side=left}{vpath(__v1,__v2)}
				\fmfi{phantom,label=$W^{\pm}$,label.side=left}{vpath(__v4,__v2)}
				\fmfi{phantom,label=$H^{\pm}$,label.side=left}{vpath(__v3,__v4)}
			\end{fmfgraph*}
		\end{fmffile}
		\caption{Charged Higgs, charged Higgs loop}
		\label{fig:BarrZee-cHW-cHloop}
	\end{subfigure}
	\caption{Specific Two-loop Barr-Zee diagrams}
	\label{fig:BarrZee-specific}
\end{figure}
\begin{figure}[p]
	\centering
	\begin{subfigure}[t]{0.45\textwidth}
		\centering
		\begin{fmffile}{BarrZee-chromo}
			\begin{fmfgraph*}(180, 120)
				\fmfleft{i1,i2}
		   		\fmfright{o1,o2}
		   		\fmf{plain}{i1,v1}
		   		\fmf{fermion,tension=0.25}{v1,v2}
		   		\fmf{plain}{v2,o1}
				\fmffreeze
				\fmfpoly{phantom,smooth,tension=0.8}{v3,v4,v5,v6}
				\fmf{dashes}{v1,v3}
				\fmf{gluon}{v4,v2}
				\fmf{gluon,tension=1.25}{v5,o2}
				\fmf{phantom,tension=1.25}{i2,v6}
				\fmffreeze
				\fmf{plain,right=0.414}{v3,v4,v5,v6,v3}
				% \fmffreeze
				% Add our labels
				% \fmflabel{$f$}{i1}
				% \fmflabel{$f$}{o1}
				% \fmflabel{$\gamma$}{o2}
				\fmfiv{label=$f$,label.angle=60}{vloc(__i1)}
				\fmfiv{label=$f$,label.angle=120}{vloc(__o1)}
				\fmfiv{label=$g$,label.angle=0}{vloc(__o2)}
				\fmfi{phantom,label=$\phi=h,,H,,A$,label.side=left}{vpath(__v1,__v3)}
				\fmfi{phantom,label=$g$,label.side=left}{vpath(__v4,__v2)}
				\fmfi{phantom,label=$t$,label.side=left}{vpath(__v3,__v4)}
			\end{fmfgraph*}
		\end{fmffile}
		\caption{Barr-Zee diagram, gluons}
		\label{fig:BarrZee-chromo}
	\end{subfigure}
	\begin{subfigure}[t]{0.45\textwidth}
		\begin{fmffile}{Weinberg}
			\begin{fmfgraph*}(180, 120)
				\fmfleft{i1}
				\fmfright{o1,o2}
				\fmf{gluon,tension=1.25}{i1,v1}
				\fmfpolyn{phantom,smooth,tension=0.8}{v}{8}
				\fmf{gluon}{v4,o1}
				\fmf{gluon}{v6,o2}
				\fmffreeze
				\fmf{dashes}{v3,v7}
				\fmfcyclen{plain,right=0.199}{v}{8}
				% Add our labels
				\fmfiv{label=$g$,label.angle=60}{vloc(__i1)}
				\fmfiv{label=$g$,label.angle=60}{vloc(__o1)}
				\fmfiv{label=$g$,label.angle=-60}{vloc(__o2)}
				\fmfi{phantom,label=$\phi/H^{\pm}$,label.side=right}{vpath(__v3,__v7)}
				\fmfiv{label=$t$,label.angle=0}{vloc(__v1)}
				\fmfiv{label=$t/b$,label.angle=0}{vloc(__v5)}
				% \fmfi{phantom,label=$t$,label.side=right}{vpath(__v8,__v2)}
				% \fmfi{phantom,label=$t(b)$,label.side=right}{vpath(__v4,__v6)}
			\end{fmfgraph*}
		\end{fmffile}
		\caption{Weinberg diagram}
		\label{fig:Weinberg}
	\end{subfigure}
	\caption{Diagrams relevant to chromo-EDM}
	\label{fig:chromo-edm-all}
\end{figure}