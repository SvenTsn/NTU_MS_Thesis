\chapter{Electric Dipole Moment}
\label{ch:EDM}

\section{The electric dipole moment in particle physics}
Classically, the electric dipole moment of a charge distribution is found as the coefficient of the \(l = 1 \) term in the multipole expansion of the potential of said distribution.
Explicitly, it is\footnote{In classical texts, the dipole moment is represented with the symbol \(\vb{p} \).
For the sake of consistency with the particle physics notation, I have chosen the symbol \(\vb{d} \) instead.}
\begin{equation}
	\vb{d} = \int (\vb{r}-\vb{r}_{0})\rho(\vb{r})\dd[3]{\vb{r}}
\end{equation}
with \(\vb{r}_{0} \) the center of mass of the charge distribution~\cite{Griffiths2013Electrodynamics, Jackson1999Electrodynamics}.
It is related to the seperation of charges or nonuniformity along a single axis in the charge distribution;
for a distribution of nonzero net charge, it is also related to the position of the center of mass with respect to the origin.
It is inherently a property of systems of finite size\footnote{There are point-like ``dipoles'' etc. in the classical framework, but they are purely mathematical and are not attributed to any physical thing.}.
However, in particle physics, we are not dealing with composite systems, but the properties of the fundamental particles.
That means we are instead looking for an \textit{intrinsic} EDM of point-like particles that is attributed to the particle itself, and not a seperation of charges.
This is analogous to the (spin) magnetic dipole moment being an \textit{intrinsic} property and not a tiny loop of current.
In particluar, particle physicists are interested in the EDMs of various fermions, since they are the point-like building blocks of all matter.
Generally speaking, there are two ways to introduce this ``EDM term'', approaching the issue from slightly different angles.

\subsection{The Dirac method}
The first method is that of Dirac. 
In his 1928 paper~\cite{Dirac1928DiracEquation}, which gave birth to his famous Dirac equation, he also wrote down the equation for the electron in an arbitrary electromagnetic field.
Adapted into more modern notation, it is written as
\begin{equation}\label{eq:DiracEqwithEMField}
	\left[-(E-e\phi) + \bm{\alpha}\vdot(\vb{p}-e\vb{A}) + \beta m\right]\psi = 0
\end{equation}
where
\begin{equation}
	\alpha_{i} = \gamma^{5}\Sigma_{i} = \begin{pmatrix} 0 & I \\ I & 0 \end{pmatrix}\begin{pmatrix} \sigma_{i} & 0 \\ 0 & \sigma_{i} \end{pmatrix}= \begin{pmatrix} 0 & \sigma_{i} \\ \sigma_{i} & 0 \end{pmatrix},
	\quad \beta = \begin{pmatrix} I & 0 \\ 0 & -I \end{pmatrix}
\end{equation}
in the Pauli-Dirac representation.
By ``squaring'' \eqnref{DiracEqwithEMField} one obtains
\begin{align}
	&\left[(E-e\phi) + \bm{\alpha}\vdot(\vb{p}-e\vb{A}) + \beta m\right]\left[-(E-e\phi) + \bm{\alpha}\vdot(\vb{p}-e\vb{A}) + \beta m\right]\psi = 0 \nonumber \\
	&\Longrightarrow \left[-(E-e\phi)^{2} + (\bm{\Sigma}\vdot(\vb{p}-e\vb{A}))^{2} + m^{2} \right. \nonumber \\
	&\qquad \left. + \gamma^{5}\left\{(E-e\phi)(\bm{\Sigma}\vdot(\vb{p}-e\vb{A})) - (\bm{\Sigma}\vdot(\vb{p}-e\vb{A}))(E-e\phi)\right\}\right]\psi = 0
\end{align}
After some matrix algebra, in particular \((\bm{\Sigma}\vdot\vb{a})(\bm{\Sigma}\vdot\vb{b}) = (\vb{a}\vdot\vb{b})I + i\bm{\Sigma}\vdot(\vb{\vb{a}\cross\vb{b}}) \), and casting \(E \) and \(\vb{p} \) back into their operator forms when necessary, we arrive at
\begin{equation}
	\left[-(E-e\phi)^{2} + (\vb{p}-e\vb{A})^{2} + m^{2} - e\bm{\Sigma}\vdot\vb{B} - ie\gamma^{5}\bm{\Sigma}\vdot\vb{E}\right]\psi = 0
\end{equation}
which introduces two additional terms, corresponding to the magnetic and the electric dipole moment of the electron, respectively.
At the time, Dirac thought the EDM term was merely a consequence of the ``squaring'' used in the derivation
arbitrarily introducing an imaginary term from a real Hamiltonian, and disregarded it.
It was only until 1958 when Salpeter~\cite{Salpeter1958EDMTerm} reintroduced this term in an ``interaction Lagrangian'' point of view.

\subsection{Form factor decomposition}
The second method is based on form factor decomposition. This method was recently seen in~\cite{Nowakowski2005FormFactor}.
We start by expressing the expectation value of the electromagnetic 4-current in momentum space as 
\begin{equation}
	\mel{p_{f}}{j^{\mu}}{p_{i}} = \bar{u}(\vb{p_{f}})\mathcal{O}^{\mu}(l,q)u(\vb{p_{i}})
\end{equation}
where \(l \equiv p_{f} + p_{i} \) is the total 4-momentum, \(q \equiv p_{f} - p_{i}\) is the momentum transfer,
and \(\mathcal{O}^{\mu}(l,q) \) is an operator whose matrix element between the spinors is a Lorentz vector.
We then want to decompose \(\mathcal{O}^{\mu}(l,q) \) in terms of the Clifford algebra of the Dirac gamma matrices.
After identifying all possible combinations and contractions of \(\left\{q^{\mu}, l^{\mu}, \gamma^{\mu}, \gamma^{5}, \sigma^{\mu\nu}, \epsilon^{\mu\nu\alpha\beta}\right\} \),
applying various Gordon-like identities, and imposing gauge invariance\footnote{Note that without gauge invariance imposed, there ought to be six independent terms, which is the case for the Weak current.} \(q_{\mu}j^{\mu} = 0 \), we arrive at four independent terms
\begin{align}
	&\bar{u}(\vb{p_{f}})\mathcal{O}^{\mu}(l,q)u(\vb{p_{i}}) \nonumber \\
	&= \bar{u}(\vb{p_{f}})\left\{F_{1}(q^{2})\gamma^{\mu} + \frac{i\sigma^{\mu\nu}}{2m}q_{\nu}F_{2}(q^{2})\right. \nonumber \\
	&\left.\qquad\quad + (-\gamma^{5}\sigma^{\mu\nu}q_{\nu})\frac{1}{4m}F_{3}(q^{2})
	+ \frac{1}{2m}\left(q^{\mu}-\frac{q^{2}}{2m}\gamma^{\mu}\gamma^{5}\right)F_{4}(q^{2})\right\}u(\vb{p_{i}})
\end{align}

We then couple this with the electromagnetic potential \(A_{\mu} \) to obtain some physical insight on this result.
Taking the non-relativistic limit \((q^{2} = 0) \), we can identify 
\begin{equation}
	F_{1}(0) = Q, \quad \frac{1}{2m}\left(F_{1}(0)+F_{2}(2)\right) = \mu, \quad -\frac{1}{2m}F_{3}(0) = d
\end{equation}
are the charge, magnetic dipole moment, and electric dipole moment, respectively.
Transforming the coupled current plus EM field back into position space, we can obtain the addition to the interaction Lagranginan for each term as well.

\subsection{General properties of EDM}
Regardless of approach, the end result, in terms of quantum field theory, is the inclusion of an effective dimension-5 interaction operator to the EM Lagrangian,
\begin{equation}\label{eq:edm_interaction-term}
  -\frac{i}{2}d_{f}\left(\bar{f}\sigma^{\mu\nu}\gamma_{5}f\right)F_{\mu\nu}.
\end{equation}
that produces EDM \(d_{f} \) for a fermion \(f \), where \(F_{\mu\nu} \) is the electromagnetic field strength tensor.
If we examine the EDM term under the lens of discrete symmetry transformations, we can see that
\begin{equation}
\begin{array}{rclcrcl}
	\vb{E} & \xrightarrow{P} & -\vb{E} & & \vb{E} & \xrightarrow{T} & \vb{E} \\
	\bm{\Sigma} & \xrightarrow{P} & \bm{\Sigma} & & \bm{\Sigma} & \xrightarrow{T} & -\bm{\Sigma}
\end{array}
\end{equation}
for parity transformation \(P\) and time-reversal transformation \(T\).
Thus, a nonzero EDM of fundamental particles indicates that both time-reversal and parity invariance are broken.

\section{G2HDM contributions to EDMs}\label{sec:g2hdm_in_edm}
\subsection{One-loop v.s. Two-loop}
In {\gthdm}, the first finite contribution to EDM appears at one-loop (\figref{oneloop}).
% Figure: General one-loop diagram
If we look at the figure, there are two Yukawa interaction vertices along the fermion line.
Each Yukawa interaction vertex flips the chirality of the fermion.
In the previous section we observed that the dipole operator breaks parity invariance, and thus is chirality violating.
Since the only chirality-flipping vertices along the fermion line are the Yukawa interaction vertices,
we need an extra chirality flip somewhere along the line to obtain the correct chiral structure.
This is achieved by requiring an additional mass insertion on the fermion line, i.e. treating the small mass of the fermion as an ``interaction'' between the (massless) chiral states.
Since the mass of the fermions in question are very small, both the Yukawa vertices and the mass insertion effectively suppress the amplitude of the one loop-diagrams.
This is known as ``chiral suppression'', and opens the possibility of two-loop diagrams making meaningful contributions.

A certain set of two-loop diagrams, known as Barr-Zee diagrams (\figref{BarrZee_general}), are indeed of importance.
% Figure: General Barr-Zee diagram 
These diagrams were first mentioned by Bjorken and Weinberg~\cite{BjorkenWeinberg1977TwoLoop},
but it was Barr and Zee~\cite{BarrZee1990TwoLoop} that calculated neutral scalar contributions with top quark and gauge bosons in the loop.
This was later extended~\cite{Leigh1991EDM,Chang1991EDM,Kao1992EDM,BowserChao1997EDM,Abe2016EDM} to include other two-loop diagrams.
These diagrams only contain one Yukawa interaction vertex, hence one chirality flip, as opposed to three in the one-loop case.
The relative enhancement due to the lack of the two mass factors more than compensate for the suppression from the additional loop factor.
Thus, interestingly, the two-loop diagrams become the dominant contribution to the EDM interaction operator.

\subsection{Two-loop Formulae}
In this subsection, we list all the relevant formulae involved in the calculation of the two-loop Barr-Zee diagram amplitudes with different loop particles.
The formulae are compiled from various resources, and verified by us before using them.
We also explicitly draw out all of the corresponding diagrams in~\figref{BarrZee-specific}.
The formulae for the neutral-Higgs-mediated diagrams are taken from~\cite{Abe2016EDM};
The formulae for the charged-Higgs-mediated diagrams are taken from~\cite{BowserChao1997EDM,Abe2016EDM}.
The notation follows that of~\cite{Abe2016EDM}.
\begin{align}\label{eq:BarrZee-phiG-toploop}
	(d^{\phi G}_{l})_{t} = -&\frac{e\,m_{l}}{(4\pi)^{4}}\sqrt{2}G_{F}\sum_{\phi=h,H,A}\sum_{G=\gamma,Z}N_{c}Q_{t}(g_{Gll}^{L}+g_{Gll}^{R}) \nonumber \\
	& \times \left[\frac{g_{\phi ll}^{A}}{m_{l}/v}\frac{g_{\phi tt}^{V}}{m_{t}/v}\mathcal{I}_{1}^{G}(m_{t}, m_{\phi}) + \frac{g_{\phi ll}^{V}}{m_{l}/v}\frac{g_{\phi tt}^{A}}{m_{t}/v}\mathcal{I}_{2}^{G}(m_{t}, m_{\phi})\right]
\end{align}
where
\begin{align}
	\mathcal{I}_{1}^{G}(m_{t},m_{\phi}) = (g_{Gtt}^{L}+g_{Gtt}^{A})\frac{m_{t}^{2}}{m_{\phi}^{2}-m_{G}^{2}}
	\left(-2\frac{m_{G}^{2}}{m_{t}^{2}} f\left(\frac{m_{t}^{2}}{m_{G}^{2}}\right) + 2\frac{m_{\phi}^{2}}{m_{t}^{2}} f\left(\frac{m_{t}^{2}}{m_{\phi}^{2}}\right)\right) \nonumber \\
	\mathcal{I}_{2}^{G}(m_{t},m_{\phi}) = (g_{Gtt}^{L}+g_{Gtt}^{A})\frac{m_{t}^{2}}{m_{\phi}^{2}-m_{G}^{2}}
	\left(-2\frac{m_{G}^{2}}{m_{t}^{2}} g\left(\frac{m_{t}^{2}}{m_{G}^{2}}\right) + 2\frac{m_{\phi}^{2}}{m_{t}^{2}} g\left(\frac{m_{t}^{2}}{m_{\phi}^{2}}\right)\right)
\end{align}

\begin{equation}\label{eq:BarrZee-phiG-Wloop}
	(d^{\phi G}_{l})_{W} = +\frac{e\,m_{l}}{(4\pi)^{4}}\sqrt{2}G_{F}\sum_{\phi=h,H,A}\sum_{G=\gamma,Z}(g_{Gll}^{L}+g_{Gll}^{R})\frac{g_{\phi ll}^{A}}{m_{l}/v}\frac{g_{WW\phi}}{2m_{W}^{2}/v}\mathcal{I}_{W}^{G}(m_{\phi})
\end{equation}
where
\begin{adjustwidth}{-2.5cm}{-2.5cm}
	\begin{align}
	\mathcal{I}_{W}^{G}(m_{\phi}) = &g_{WWG}\frac{2m_{W}^{2}}{m_{\phi}^{2}-m_{G}^{2}} \nonumber \\
	& \times \left[-\frac{1}{4}\left\{\left(6-\frac{m_{G}^{2}}{m_{W}^{2}}\right) + \left(1-\frac{m_{G}^{2}}{2m_{W}^{2}}\right)\frac{m_{\phi}^{2}}{m_{W}^{2}}\right\}
	\left[-2\frac{m_{\phi}^{2}}{m_{W}^{2}} f\left(\frac{m_{W}^{2}}{m_{\phi}^{2}}\right) + 2\frac{m_{G}^{2}}{m_{W}^{2}} f\left(\frac{m_{W}^{2}}{m_{G}^{2}}\right)\right]\right. \nonumber \\
	&\left. \quad + \left\{\left(-4+\frac{m_{G}^{2}}{m_{W}^{2}}\right) + \frac{1}{4}\left(\left(6-\frac{m_{G}^{2}}{m_{W}^{2}}\right) + \left(1-\frac{m_{G}^{2}}{2m_{W}^{2}}\right)\frac{m_{\phi}^{2}}{m_{W}^{2}}\right)\right\}
	\left[-2\frac{m_{\phi}^{2}}{m_{W}^{2}} g\left(\frac{m_{W}^{2}}{m_{\phi}^{2}}\right) + 2\frac{m_{G}^{2}}{m_{W}^{2}} g\left(\frac{m_{W}^{2}}{m_{G}^{2}}\right)\right]\right]
	\end{align}
\end{adjustwidth}

\begin{equation}\label{eq:BarrZee-phiG-cHloop}
	(d^{\phi G}_{l})_{H^{\pm}} = +\frac{e\,m_{l}}{(4\pi)^{4}}\sqrt{2}G_{F}\sum_{\phi=h,H,A}\sum_{G=\gamma,Z}(g_{Gll}^{L}+g_{Gll}^{R})\frac{g_{\phi ll}^{A}}{m_{l}/v}\frac{g_{\phi H^{+}H^{-}}}{v}\mathcal{I}_{3}^{G}(m_{H^{\pm}}, m_{\phi})
\end{equation}
where
\begin{adjustwidth}{-2.5cm}{-2.5cm}
	\begin{align}
	\mathcal{I}_{3}^{G}(m_{H^{\pm}}, m_{\phi}) =& -\frac{1}{2}g_{GH^{+}H^{-}}\frac{v^{2}}{m_{\phi}^{2}-m_{G}^{2}} \nonumber \\
	& \times \left[\left(-2\frac{m_{G}^{2}}{m_{H^{\pm}}^{2}} f\left(\frac{m_{H^{\pm}}^{2}}{m_{G}^{2}}\right) + 2\frac{m_{\phi}^{2}}{m_{H^{\pm}}^{2}} f\left(\frac{m_{H^{\pm}}^{2}}{m_{\phi}^{2}}\right)\right)
	-\left(-2\frac{m_{G}^{2}}{m_{H^{\pm}}^{2}} g\left(\frac{m_{H^{\pm}}^{2}}{m_{G}^{2}}\right) + 2\frac{m_{\phi}^{2}}{m_{H^{\pm}}^{2}} g\left(\frac{m_{H^{\pm}}^{2}}{m_{\phi}^{2}}\right)\right)\right]
	\end{align}
\end{adjustwidth}

\begin{equation}\label{eq:BarrZee-cHW-tbloop}
	(d^{H^{+}W^{+}}_{l})_{t/b} = -\frac{3e m_{l}}{(4\pi)^{4}}\frac{m_{w}^{2}}{v^{4}}
	\left(\frac{v^{2}}{2m_{t}m_{l}}\Im{\rho_{tt}\rho_{ll}}\big(Q_{t}F_{t}(m_{H^{\pm}})+Q_{b}F_{b}(m_{H^{\pm}})\big)\right)
\end{equation}
where
\begin{equation}
	F_{q}(m_{H^{\pm}}) = \frac{T_{q}(m_{H^{\pm}}^{2}/m_{t}^{2}) - T_{q}(m_{W}^{2}/m_{t}^{2})}{(m_{H^{\pm}}^{2}/m_{t}^{2}) - (m_{W}^{2}/m_{t}^{2})}
\end{equation}

\begin{equation}\label{eq:BarrZee-cHW-Wloop}
	(d^{H^{+}W^{+}}_{l})_{W} = -\frac{e m_{l}}{(4\pi)^{4}}\sqrt{2}G_{F}\mathcal{S}_{l}\sum_{h}
	\frac{g_{llh}^{A}}{m_{l}/v}\frac{g_{WWh}}{2m_{W}^{2}/v}\frac{e^{2}}{2s_{W}^{2}}\mathcal{I}_{4}(m_{h}, m_{H^{\pm}})
\end{equation}
where
\begin{equation}
	\mathcal{I}_{4}(m_{h}, m_{H^{\pm}}) = \frac{m_{W}^{2}}{m_{H^{\pm}}^{2}-m_{W}^{2}}\big[I_{4}(m_{W}, m_{h}) - I_{4}(m_{H^{\pm}}, m_{h})\big]
\end{equation}

\begin{equation}\label{eq:BarrZee-cHW-cHloop}
	(d^{H^{+}W^{+}}_{l})_{H^{+}} = -\frac{e m_{l}}{(4\pi)^{4}}\sqrt{2}G_{F}\mathcal{S}_{l}\sum_{h}
	\frac{g_{llh}^{A}}{m_{l}/v}\frac{g_{HHh}}{v}\frac{e^{2}}{2s_{W}^{2}}\mathcal{I}_{5}(m_{h}, m_{H^{\pm}})
\end{equation}
where
\begin{equation}
	\mathcal{I}_{5}(m_{h}, m_{H^{\pm}}) = \frac{m_{W}^{2}}{m_{H^{\pm}}^{2}-m_{W}^{2}}\big[I_{5}(m_{W}, m_{h}) - I_{5}(m_{H^{\pm}}, m_{h})\big]
\end{equation}

with loop functions
\begin{align}
	f(a) &= \frac{1}{2} a \int_{0}^{1}\dd{z}\frac{1-2z(1-z)}{z(1-z)-a}\log{\frac{z(1-z)}{a}} \\
	g(a) &= \frac{1}{2} a \int_{0}^{1}\dd{z}\frac{1}{z(1-z)-a}\log{\frac{z(1-z)}{a}}
\end{align}

\begin{adjustwidth}{-1.5cm}{-1.5cm}
	\begin{align}
	T_{t}(a) &= \frac{1-3a}{a^{2}}\frac{\pi^{2}}{6} + \left(\frac{1}{a}-\frac{5}{2}\right)\log{a}
	- \frac{1}{a} - \left(1-\frac{1}{a}\right)\left(2-\frac{1}{a}\right)\dilog(1-a) \\
	T_{b}(a) &= \frac{2a-1}{a^{2}}\frac{\pi^{2}}{6} + \left(\frac{3}{2}-\frac{1}{a}\right)\log{a}
	+ \frac{1}{a} - \frac{1}{a}\left(1-\frac{1}{a}\right)\dilog(1-a)
	\end{align}
\end{adjustwidth}

\begin{adjustwidth}{-2.5cm}{-2.5cm}
	\begin{align}
	I_{4}(m_{i}, m_{\phi}) &= \int_0^1 d z \frac{m_i^2\left(z(1-z)^2-4(1-z)^2+\frac{m_{H^{\pm}}^{2}m_\phi^2}{m_W^2} z(1-z)^2\right)}{m_W^2(1-z)+m_\phi^2 z-m_i^2 z(1-z)} \log \left(\frac{m_W^2(1-z)+m_\phi^2 z}{m_i^2 z(1-z)}\right)\\
	I_{5}(m_{i}, m_{\phi}) &= 2 \int_0^1 d z \frac{m_i^2 z(1-z)^2}{m_{H^{\pm}}^2(1-z)+m_\phi^2 z-m_i^2 z(1-z)} \log \left(\frac{m_{H^{\pm}}^2(1-z)+m_\phi^2 z}{m_i^2 z(1-z)}\right)
	\end{align}
\end{adjustwidth}

\section{Chromoelectric dipole moment}
If we were only interested in lepton EDM, then the previous discussion would have been sufficient, since leptons only participate in the electroweak interaction.
However, quarks also participate in the strong interaction, so there will be QCD-related effects.
This can be found in two additional terms in the Lagrangian: 
the chromo-EDM \(\tilde{d}_{f} \) for fermion \(f \), and the Weinberg term \(C_{W} \) for gluon interactions~\cite{Weinberg1989Gluon}, written as
\begin{equation}
  -\frac{i g_{s}}{2}\tilde{d_{f}}\left(\bar{f}\sigma^{\mu\nu}T^{a}\gamma_{5}f\right)G^{a}_{\mu\nu}\quad \qquad \quad -\frac{1}{3}C_Wf^{abc}G^{a}_{\mu\sigma}G^{b,\sigma}_{\nu}\tilde{G}^{c,\mu\nu}
\end{equation}
which correspond to the diagrams in~\figref{chromo-edm-all}.
The chromo-EDM term is essentially when the vector bosons of the ``ordinary'' EDM interaction are replaced by gluons instead.
Since only quarks interact with gluons, only the diagram with a fermion loop remains.
The Weinberg term is relevant because quarks always exist in bound hadronic states.
That means experimentally we use the EDM of hadrons to infer about quark EDM, and thus the gluon interactions within the hadrons will contribute to the hadron EDM.
We evaluate the contributions to \(\tilde{d}_{u, d} \) and \(C_{W} \) in {\gthdm} by following Refs.~\cite{Abe14} and~\cite{JungPich14}, with discussion on theoretical uncertainties found in Ref.~\cite{KanetaEtAl23}.
The formulae for calculating the cEDM are

\begin{adjustwidth}{-1.5cm}{-1.5cm}
	\begin{equation}
	\tilde{d_{f}} = +\frac{m_{f}}{(4\pi)^{4}}\sqrt{2}G_{F}\sum_{\phi=h,H,A}2g_{s}^{2}\frac{m_{f}^{2}}{m_{\phi}^{2}}
	\left[\frac{g_{\phi ff}^{A}}{m_{f}/v}\frac{g_{\phi tt}^{V}}{m_{t}/v}\left(-2\frac{m_{\phi}^{2}}{m_{t}^{2}} f\left(\frac{m_{t}^{2}}{m_{\phi}^{2}}\right)\right) 
	+ \frac{g_{\phi ff}^{V}}{m_{f}/v}\frac{g_{\phi tt}^{A}}{m_{t}/v}\left(-2\frac{m_{\phi}^{2}}{m_{t}^{2}} g\left(\frac{m_{t}^{2}}{m_{\phi}^{2}}\right)\right)\right]
	\end{equation}
\end{adjustwidth}

\begin{equation}
	C_{W} = 
\end{equation}

The contribution of the Weinberg diagram can be evaluated using QCD running.

\begin{align}
	\frac{d_{f}(\mu_{h})}{2} &= \\
	\frac{\tilde{d_{f}}(\mu_{h})}{2} &= \\
	C_{W}(\mu_{h}) &= 
\end{align}

where constants.

\clearpage
% General One-loop diagram
\begin{figure}[p]
	\centering
	\begin{fmfgraph*}(300, 125)
		\fmfleft{i1,i2}
		\fmfright{o1,o2}
		\fmf{plain}{i1,v1}
		\fmf{dashes,tension=0.3}{v1,v2}
		\fmf{plain}{v2,o1}
		\fmffreeze
		\fmf{photon,tension=1}{v4,o2}
		\fmf{phantom,tension=1}{i2,v3}
		\fmf{plain,left=0.268}{v1,v3}
		\fmf{plain,tension=1.65,left=0.268}{v3,v4}
		\fmf{plain,left=0.268}{v4,v2}
		\fmfiv{label=$f$,label.angle=60}{vloc(__i1)}
		\fmfiv{label=$f$,label.angle=120}{vloc(__o1)}
		\fmfiv{label=$\gamma$,label.angle=0}{vloc(__o2)}
		\fmfi{phantom,label=$\phi$,label.side=left}{vpath(__v1,__v2)}
		% \fmfi{phantom,label=$G=\gamma,,Z$,label.side=left}{vpath(__v4,__v2)}
		% \fmfi{phantom,label=$t$,label.side=left}{vpath(__v3,__v4)}
	\end{fmfgraph*}
	\caption{One-loop diagram}
	\label{fig:oneloop}
\end{figure}

% General Barr-Zee diagram
\begin{figure}[p]
	\centering
	\begin{fmfgraph*}(300, 200)
		\fmfleft{i1,i2}
		\fmfright{o1,o2}
		\fmf{plain}{i1,v1}
		\fmf{fermion,tension=0.25}{v1,v2}
		\fmf{plain}{v2,o1}
		\fmffreeze
		\fmfpoly{plain,smooth,filled=shaded,tension=0.8}{v3,v4,v5,v6}
		\fmf{dashes}{v1,v3}
		\fmf{boson}{v4,v2}
		\fmf{photon,tension=1.25}{v5,o2}
		\fmf{phantom,tension=1.25}{i2,v6}
		\fmffreeze

		\fmfiv{label=$f$,label.angle=60}{vloc(__i1)}
		\fmfiv{label=$f$,label.angle=120}{vloc(__o1)}
		\fmfiv{label=$\gamma$,label.angle=0}{vloc(__o2)}
		\fmfi{phantom,label=$\phi$,label.side=left}{vpath(__v1,__v3)}
		\fmfi{phantom,label=$V$,label.side=left}{vpath(__v4,__v2)}
		% \fmfi{phantom,label=$f$,label.side=right,label.dist=0.01w}{vpath(__v3,__v4)}
	\end{fmfgraph*}
	\caption{Two-loop Barr-Zee diagram}
	\label{fig:BarrZee_general}
\end{figure}

% Specific Barr-Zee diagrams
\begin{figure}[p]
	\centering
	\begin{adjustbox}{center}
	\begin{subfigure}[t]{0.6\textwidth}
		\centering
		\barrzeespecific{(210,140)}{neutralHiggs}{toploop}
		\caption{Neutral Higgs, top loop}
		\label{fig:BarrZee-phiG-toploop}
	\end{subfigure}\hfill
	\begin{subfigure}[t]{0.6\textwidth}
		\centering
		\barrzeespecific{(210,140)}{neutralHiggs}{Wloop}
		\caption{Neutral Higgs, \(W \) loop}
		\label{fig:BarrZee-phiG-Wloop}
	\end{subfigure}
	\end{adjustbox}
	\begin{adjustbox}{center}
	\begin{subfigure}[t]{0.6\textwidth}
		\centering
		\barrzeespecific{(210,140)}{neutralHiggs}{cHloop}
		\caption{Neutral Higgs, charged Higgs loop}
		\label{fig:BarrZee-phiG-cHloop}
	\end{subfigure}\hfill
	\begin{subfigure}[t]{0.6\textwidth}
		\centering
		\barrzeespecific{(210,140)}{chargedHiggs}{toploop}
		\caption{Charged Higgs, top/bottom loop}
		\label{fig:BarrZee-cHW-tbloop}
	\end{subfigure}
	\end{adjustbox}
	\begin{adjustbox}{center}
	\begin{subfigure}[t]{0.6\textwidth}
		\centering
		\barrzeespecific{(210,140)}{chargedHiggs}{Wloop}
		\caption{Charged Higgs, \(W \) loop}
		\label{fig:BarrZee_cHW_Wloop}
	\end{subfigure}\hfill
	\begin{subfigure}[t]{0.6\textwidth}
		\centering
		\barrzeespecific{(210,140)}{chargedHiggs}{cHloop}
		\caption{Charged Higgs, charged Higgs loop}
		\label{fig:BarrZee-cHW-cHloop}
	\end{subfigure}
	\end{adjustbox}
	\caption{Specific Two-loop Barr-Zee diagrams}
	\label{fig:BarrZee-specific}
\end{figure}

% Chromo-related diagrams
\begin{figure}[p]
	\centering
    \begin{adjustbox}{center}
	\begin{subfigure}[t]{0.6\textwidth}
		\centering
		\barrzeespecific[chromo]{(210,140)}{neutralHiggs}{toploop}
		\caption{Barr-Zee diagram, gluons}
		\label{fig:BarrZee-chromo}
	\end{subfigure}\hfill
	\begin{subfigure}[t]{0.6\textwidth}
		\begin{fmfgraph*}(180, 120)
			\fmfleft{i1}
			\fmfright{o1,o2}
			\fmf{gluon,tension=1.25}{i1,v1}
			\fmfpolyn{phantom,smooth,tension=0.8}{v}{8}
			\fmf{gluon}{v4,o1}
			\fmf{gluon}{v6,o2}
			\fmffreeze
			\fmf{dashes}{v3,v7}
			\fmfcyclen{plain,right=0.199}{v}{8}
			% Add our labels
			\fmfiv{label=$g$,label.angle=60}{vloc(__i1)}
			\fmfiv{label=$g$,label.angle=60}{vloc(__o1)}
			\fmfiv{label=$g$,label.angle=-60}{vloc(__o2)}
			\fmfi{phantom,label=$\phi/H^{\pm}$,label.side=right}{vpath(__v3,__v7)}
			\fmfiv{label=$t$,label.angle=0}{vloc(__v1)}
			\fmfiv{label=$t/b$,label.angle=0}{vloc(__v5)}
			% \fmfi{phantom,label=$t$,label.side=right}{vpath(__v8,__v2)}
			% \fmfi{phantom,label=$t(b)$,label.side=right}{vpath(__v4,__v6)}
		\end{fmfgraph*}
		\caption{Weinberg diagram}
		\label{fig:Weinberg}
	\end{subfigure}
    \end{adjustbox}
	\caption{Diagrams relevant to chromo-EDM}
	\label{fig:chromo-edm-all}
\end{figure}