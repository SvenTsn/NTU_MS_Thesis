% General One-loop diagram
\begin{figure}[p]
	\centering
	\begin{fmfgraph*}(300, 125)
		\fmfleft{i1,i2}
		\fmfright{o1,o2}
		\fmf{plain}{i1,v1}
		\fmf{dashes,tension=0.3}{v1,v2}
		\fmf{plain}{v2,o1}
		\fmffreeze
		\fmf{photon,tension=1}{v4,o2}
		\fmf{phantom,tension=1}{i2,v3}
		\fmf{plain,left=0.268}{v1,v3}
		\fmf{plain,tension=1.65,left=0.268}{v3,v4}
		\fmf{plain,left=0.268}{v4,v2}
		\fmfiv{label=$f$,label.angle=60}{vloc(__i1)}
		\fmfiv{label=$f$,label.angle=120}{vloc(__o1)}
		\fmfiv{label=$\gamma$,label.angle=0}{vloc(__o2)}
		\fmfi{phantom,label=$\phi$,label.side=left}{vpath(__v1,__v2)}
		% \fmfi{phantom,label=$G=\gamma,,Z$,label.side=left}{vpath(__v4,__v2)}
		% \fmfi{phantom,label=$t$,label.side=left}{vpath(__v3,__v4)}
	\end{fmfgraph*}
	\caption{One-loop diagram}
	\label{fig:oneloop}
\end{figure}

% General Barr-Zee diagram
\begin{figure}[p]
	\centering
	\begin{fmfgraph*}(300, 200)
		\fmfleft{i1,i2}
		\fmfright{o1,o2}
		\fmf{plain}{i1,v1}
		\fmf{fermion,tension=0.25}{v1,v2}
		\fmf{plain}{v2,o1}
		\fmffreeze
		\fmfpoly{plain,smooth,filled=shaded,tension=0.8}{v3,v4,v5,v6}
		\fmf{dashes}{v1,v3}
		\fmf{boson}{v4,v2}
		\fmf{photon,tension=1.25}{v5,o2}
		\fmf{phantom,tension=1.25}{i2,v6}
		\fmffreeze

		\fmfiv{label=$f$,label.angle=60}{vloc(__i1)}
		\fmfiv{label=$f$,label.angle=120}{vloc(__o1)}
		\fmfiv{label=$\gamma$,label.angle=0}{vloc(__o2)}
		\fmfi{phantom,label=$\phi$,label.side=left}{vpath(__v1,__v3)}
		\fmfi{phantom,label=$V$,label.side=left}{vpath(__v4,__v2)}
		% \fmfi{phantom,label=$f$,label.side=right,label.dist=0.01w}{vpath(__v3,__v4)}
	\end{fmfgraph*}
	\caption{Two-loop Barr-Zee diagram}
	\label{fig:BarrZee_general}
\end{figure}

% Specific Barr-Zee diagrams
\begin{figure}[p]
	\centering
	\begin{adjustbox}{center}
	\begin{subfigure}[t]{0.6\textwidth}
		\centering
		\barrzeespecific{(210,140)}{neutralHiggs}{toploop}
		\caption{Neutral Higgs, top loop}
		\label{fig:BarrZee-phiG-toploop}
	\end{subfigure}\hfill
	\begin{subfigure}[t]{0.6\textwidth}
		\centering
		\barrzeespecific{(210,140)}{neutralHiggs}{Wloop}
		\caption{Neutral Higgs, \(W \) loop}
		\label{fig:BarrZee-phiG-Wloop}
	\end{subfigure}
	\end{adjustbox}
	\begin{adjustbox}{center}
	\begin{subfigure}[t]{0.6\textwidth}
		\centering
		\barrzeespecific{(210,140)}{neutralHiggs}{cHloop}
		\caption{Neutral Higgs, charged Higgs loop}
		\label{fig:BarrZee-phiG-cHloop}
	\end{subfigure}\hfill
	\begin{subfigure}[t]{0.6\textwidth}
		\centering
		\barrzeespecific{(210,140)}{chargedHiggs}{toploop}
		\caption{Charged Higgs, top/bottom loop}
		\label{fig:BarrZee-cHW-tbloop}
	\end{subfigure}
	\end{adjustbox}
	\begin{adjustbox}{center}
	\begin{subfigure}[t]{0.6\textwidth}
		\centering
		\barrzeespecific{(210,140)}{chargedHiggs}{Wloop}
		\caption{Charged Higgs, \(W \) loop}
		\label{fig:BarrZee_cHW_Wloop}
	\end{subfigure}\hfill
	\begin{subfigure}[t]{0.6\textwidth}
		\centering
		\barrzeespecific{(210,140)}{chargedHiggs}{cHloop}
		\caption{Charged Higgs, charged Higgs loop}
		\label{fig:BarrZee-cHW-cHloop}
	\end{subfigure}
	\end{adjustbox}
	\caption{Specific Two-loop Barr-Zee diagrams}
	\label{fig:BarrZee-specific}
\end{figure}

% Chromo-related diagrams
\begin{figure}[p]
	\centering
    \begin{adjustbox}{center}
	\begin{subfigure}[t]{0.6\textwidth}
		\centering
		\barrzeespecific[chromo]{(210,140)}{neutralHiggs}{toploop}
		\caption{Barr-Zee diagram, gluons}
		\label{fig:BarrZee-chromo}
	\end{subfigure}\hfill
	\begin{subfigure}[t]{0.6\textwidth}
		\begin{fmfgraph*}(180, 120)
			\fmfleft{i1}
			\fmfright{o1,o2}
			\fmf{gluon,tension=1.25}{i1,v1}
			\fmfpolyn{phantom,smooth,tension=0.8}{v}{8}
			\fmf{gluon}{v4,o1}
			\fmf{gluon}{v6,o2}
			\fmffreeze
			\fmf{dashes}{v3,v7}
			\fmfcyclen{plain,right=0.199}{v}{8}
			% Add our labels
			\fmfiv{label=$g$,label.angle=60}{vloc(__i1)}
			\fmfiv{label=$g$,label.angle=60}{vloc(__o1)}
			\fmfiv{label=$g$,label.angle=-60}{vloc(__o2)}
			\fmfi{phantom,label=$\phi/H^{\pm}$,label.side=right}{vpath(__v3,__v7)}
			\fmfiv{label=$t$,label.angle=0}{vloc(__v1)}
			\fmfiv{label=$t/b$,label.angle=0}{vloc(__v5)}
			% \fmfi{phantom,label=$t$,label.side=right}{vpath(__v8,__v2)}
			% \fmfi{phantom,label=$t(b)$,label.side=right}{vpath(__v4,__v6)}
		\end{fmfgraph*}
		\caption{Weinberg diagram}
		\label{fig:Weinberg}
	\end{subfigure}
    \end{adjustbox}
	\caption{Diagrams relevant to chromo-EDM}
	\label{fig:chromo-edm-all}
\end{figure}