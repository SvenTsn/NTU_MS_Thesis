\chapter{The General Two-Higgs Doublet Model}
\label{ch:g2HDM}
There have been a plethora of 2HDMs proposed throughout the years, each model with its own set of additional assumptions.
A comprehensive review of various 2HDMs can be found in~\cite{Branco20122HDMs}.

\section{The Model}
Following Gell-Mann's \textit{Totalitaian principle}, as a natural extension to the SM, we can introduce a second Higgs doublet.
This second doublet couples to all flavors and families of fermions, and has no symmetry requirement imposed upon it.
Hence, it is referred to as the "General Two Higgs Doublet Model", or {\gthdm} for short.

The {\gthdm} Lagrangian can be written as~\cite{DavidsonHaber05, HouModak21}
\begin{align}
  \mathcal{L} = - & \frac{1}{\sqrt{2}} \sum_{f = u, d, \ell} \bar f_{i} \Big[\big(-\lambda^f_i \delta_{ij} s_\gamma + \rho^f_{ij} c_\gamma\big) h
  + \big(\lambda^f_i \delta_{ij} c_\gamma + \rho^f_{ij} s_\gamma\big)H
  - i\,{\rm sgn}(Q_f) \rho^f_{ij} A\Big]  R\, f_{j} \nonumber                                                                                     \\
  -               & \bar{u}_i\left[(V\rho^d)_{ij} R-(\rho^{u\dagger}V)_{ij} L\right]d_j H^+
  - \bar{\nu}_i\rho^L_{ij} R \, \ell_j H^+ + {\rm h.c.},
  \label{eq:lagrangian}
\end{align}
where the generation indices \(i \), \(j \) are summed over, \(L \), \(R = (1\pm\gamma_{5})/2\) are projections, \(V \) is the CKM matrix for quarks and unity for leptons.
\(\lambda^f \) are the SM Yukawa matrices, and \(\rho^f \) are the extra-Yukawa matrices.
A key takeaway is that each family of fermions (u-type, d-type, lepton) is associated with its own extra-Yukawa \(\rho \) matrix.
In this scenario, flavor-changing neutral Higgs (FCNH) processes are controlled by \textit{flavor hierarchies} and \textit{alignment}.
Flavor hierarchies means that the \(\rho \) matrices somehow ``know'' the current flavor structure of the SM, represented by the ``rule of thumb''~\cite{HouKumar2020RuleOfThumb}
\begin{equation}
  \rho_{ii} \lesssim \order{\lambda_i}, \quad
  \rho_{1i} \lesssim \order{\lambda_1}, \quad
  \rho_{3j} \lesssim \order{\lambda_3},
  \label{eq:ruleofthumb}
\end{equation}
with \(j\neq 1 \).
Alignment means that \(c_{\gamma} \equiv \cos\gamma = \cos(\beta-\alpha)\) is small.
Consequently, the SM-like Higgs \(h \) is mostly controlled by the SM Yukawas, while the newly introduced \(\rho \) matrices control the exotic Higgses \(H, A, H^{\pm} \).
A remarkable feature of {\gthdm} is that \(\order{1}\,\rho_{tt}\) can drive EWBG through~\cite{FHS2018EWBGandEDM} \(\lambda_{t}\Im\rho_{tt} \).

\section{Electroweak Baryogenesis}\label{sec:EWBG}
Even though the SM Lagrangian is baryon number conserving, 
there still is the possibility of baryon-number-violating processes due to a quantum anomaly~\cite{tHooft1976BaryonNumAnomaly}.
This anomaly gives rise to baryon-number-violating but \((B-L) \) conserving sphaleron transitions~\cite{Klinkhamer1984Sphaleron}, where \(B \) is baryon number and \(L \) is lepton number.
The rate of these transitions \(\Gamma_{B} \) scale with temperature as \(\sim \exp\left[-\frac{\pi m_{W}(T)}{\alpha_{W}}\right] \)~\cite{Kuzmin1985EWBG},
with \(m_{W} \) the W boson mass and \(\alpha_{W} \) the weak coupling constant,
which means that said transitions are more rapid in the earlier symmetric phase of the universe,
before the symmetry breaking phase transition that induces the Higgs vev occurs.
If \(\Gamma_{B} \) is larger than the expansion rate of the universe, namely the Hubble parameter \(H \),
then any ``primordial'' baryon asymmetry will be washed out by these sphaleron processes.
This implies that the EWPT must be a first-order phase transition, 
and that the vev aquired in the broken phase must be large enough such that \(\Gamma_{B}^{(\mathrm{br})}(T_{C}) < H(T_{C})\) at critical temperature \(T_{C} \).
This ensures that any baryon asymmetry generated by \emph{the phase transition itself} will persist without being washed out.
During this phase transition, ``bubbles'' of the new broken phase nucleate and expand to fill the universe~\cite{Farrar1993MSMBAU}.
This creates a departure from thermal equilibrium, satisfying the third Sakharov condition.
If there exists CPV in the particle interactions that happen at the bubble walls,
then BAU can be generated.
Unfortunately, as mentioned before, the CPV present in the SM is insufficient to drive the BAU of our current observable universe.

In {\gthdm}, we have an extra Higgs doublet, as well as extra Yukawa couplings to work with.
Thermal loops of heavy Higgs bosons can create a strong enough first-order EWPT~\cite{Bochkarev1990EWBG2HDM} to satisfy the above criteria,
doing to \(\order{1} \) nondecoupled~\cite{Kanemura2005EWBGHiggs} Higgs couplings.
We can estimate BAU by~\cite{HuetNelson1996EWBG, ClineKainulainen2000EWBG}
\begin{equation}
  Y_{B} \equiv \frac{n_{B}}{s} = \frac{-3\Gamma_{B}^{(\mathrm{sym})}}{2D_{q}\lambda_{+}s}\int_{-\infty}^{0}\dd{z^{\prime}}n_{L}(z^{\prime})e^{-\lambda_{-}z^{\prime}}
\end{equation}
where \(D_{q} \simeq 8.9/T \) is the quark diffusion constant, \(s \) is the entropy density,
\(\Gamma_{B}^{\mathrm{sym}} = 120\alpha_{W}^{5}T \) is the \(B \)-changing rate in the symmetric phase,
and \(\lambda_{\pm} = [v_{w} \pm (v_{w}^{2} + 15\Gamma_{B}^{\mathrm{sym}}D_{q})^{1/2}]/2D_{q} \),
with \(v_{w} \) the bubble wall velocity.
The integration happens over \(z^{\prime} \), which is the coordinate opposite of the bubble wall expansion direction.
For \(Y_{B} \) to be nonzero, a nonvanishing total left-handed fermion number density \(n_{L} \) is needed.
The current observational value \(Y_{B}^{\mathrm{obs}} = \num{8.59e-11} \) comes from the Planck collaboration~\cite{Planck2014Yobs}.
The BAU-related CPV arises from the interaction between particles/antiparticles with the bubble wall.
As a result of their thermal motion, particles and antiparticles in the neighborhood of the bubble wall propagate through it. 
Since one side of the bubble wall has zero vev, while the other side has nonzero vev,
the particles see the bubble wall as a potential barrier and scatter from it.
A dominant process is shown in~\figref{fig:BAU-CPV_process}.
The Higgs bubble wall is denoted as \(v_{a}(x)\text{, }v_{b}(y) \), which are spacetime-dependent vevs.
The CPV source term \(S_{ij} \) for left-handed fermion \(f_{iL} \) induced by right-handed fermion \(f_{jR} \) in {\gthdm} is~\cite{FHS2018EWBGandEDM}
\begin{equation}
  S_{iLjR}(Z) = N_{C}F\Im[(Y_{1})_{ij}(Y_{2})_{ij}^{*}]v^{2}(Z)\partial_{t_{Z}}\beta(Z)
\end{equation}
where \(Z = (t_{Z}, 0, 0, z) \) is the position in the heat bath, \(N_{C} = 3 \) is the number of color,
and \(F \) is a function (see Ref.~\cite{CFS2016EWBG}).
Essentially, the CPV for BAU is contained in \(\Im[(Y_{1})_{ij}(Y_{2})_{ij}^{*}] \).
Diagonalizing gives
\begin{equation}
  \Im[(Y_{1})_{tc}(Y_{2})_{tc}^{*}] = -\lambda_{t}\Im\rho_{tt}\, \text{,} \quad \rho_{tc} = 0
\end{equation}
And thus we arrive at the most important result of~\cite{FHS2018EWBGandEDM}: BAU can be driven in the {\gthdm} by \(-\lambda_{t}\Im\rho_{tt} \).

\clearpage
\begin{figure}[p]
    \centering
    \begin{adjustbox}{center}
        \begin{fmfgraph*}(200, 125)
            \fmfleft{i1,i2}
            \fmfright{o1,o2}
            \fmf{fermion}{o1,v2,v1,i1}
            \fmf{dashes}{i2,v1}
            \fmf{dashes}{o2,v2}
            \fmffreeze
            \fmfi{phantom,label=$c_{R},,t_{R}$,label.side=right}{vpath(__v1,__v2)}
            \fmfiv{label=$v_{a}(x)$,label.angle=120}{vloc(__i2)}
            \fmfiv{label=$v_{b}(y)$,label.angle=60}{vloc(__o2)}
            \fmfiv{label=$t_{L}$,label.angle=-120}{vloc(__i1)}
            \fmfiv{label=$t_{L}$,label.angle=-60}{vloc(__o1)}
        \end{fmfgraph*}
    \end{adjustbox}
    \caption{A dominant CPV process relevant for baryon asymmetry, with Higgs bubble wall denoted \(v_{a}(x) \) and \(v_{b}(y) \).}
	\label{fig:BAU-CPV_process}
\end{figure}