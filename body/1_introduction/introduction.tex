\chapter{Introduction}
\label{ch:intro}

The Standard Model of particle physics has solidified itself over the past 50-or-so years as the fundamental theory of everything quantum.
This is both a blessing and a curse for particle physicists. 
While the SM provides a very powerful framework to explain a plethora of phenomena, 
it is so accurate in explaining what it \emph{can} explain 
yet provides next to no direction in how to probe the questions it \emph{cannot} explain.
Beyond the Standard Model (BSM) theories are a dime a dozen nowadays, 
all trying their best to answer these unanswered questions.
A question that is of particular importance is that of the baryon asymmetry of the universe (BAU),
and namely its driving process baryogenesis,
which is how the matter-antimatter imbalance came to be.
This opens the door to large BSM charge-parity violation (CPV),
since the SM CPV, which is completely housed in the CKM matrix~\cite{PDG2022},
is not enough to explain what we have observed.
However, large BSM-CPV should have led to new discoveries at the LHC, which evidently is \textit{not} what has been observed.
In this thesis, we explore CPV through a General Two Higgs doublet model ({\gthdm}), 
which is an extension of the SM by introducing an extra Higgs doublet.
We first elucidate the theortical potential of the model in generating BAU,
then turn our focus to precision experiments of the EDMs of fundamental particles.
These EDMs are very sensitive to CPV effects, so they can serve as a \emph{litmus test} of the model.
We introduce a method of satisfying the precision bounds while still maintaining adequate efficiency for BAU in {\gthdm}.
In the end, we comment on the experimental prospects for these EDM experiments,
as well as the prospects for {\gthdm} under the new and improved experiments in the coming decade or two.

\section{The Standard Model of Particle Physics}
The current working theory in the realm of particle physics is the Standard Model (SM). 
[TO BE COMPLETED]

\section{Limits of the Standard Model}
The SM is a powerful theory, providing a unifying framework for three of the four fundamental forces, and producing many verified predictions.
However, it is clear that the SM is \textit{definitely not} the ultimate framework, and there is definitely room for extensions and modifications.
To quote from the textbook \textit{Modern Particle Physics} by Mark Thomson~\cite{Thomson2013ParticleTextbook},
``[The Standard Model] is a model constructed from a number of beautiful and profound theoretical ideas
put together in a somewhat \textit{ad hoc} fashion in order to reproduce the experimental data.''
Amidst all its strength, there are still several open questions that the SM has not been able to answer.
In particular, the baryon asymmetry of the universe (BAU) problem is about the observed imbalance in baryonic matter and anti-baryonic matter amounts.
Current cosmological theory predict that the Big Bang started with an almost equal number of quarks and antiquarks~\cite{Sarkar2008AstroparticlePhysics}.
It is further hypothesized that during the expansion and cooling of the universe, some physical process occured that created the observed BAU.
This process is known as baryogenesis~\cite{Liddle2015Cosmology} (formerly baryosynthesis~\cite{BarrowTurner1981Baryosynthesis,Turner1981GUT}).
In particular, electroweak baryogenesis (EWBG) is the theory where such a baryogenesis process takes place during the electroweak phase transition (EWPT)~\cite{Kuzmin1985EWBG, Cohen1990EWBG}.
Inspired by the discoveries of the Cosmic Microwave Background (CMB)~\cite{PenziasWilson1965CMB} and CPV in the neutral Kaon system~\cite{CroninFitch1964KaonCPV},
Andrei Sakharov proposed~\cite{Sakharov1967BAU} a set of three necessary conditions for BAU, and consequentially baryogenesis, to occur:
\begin{enumerate}
  \item Baryon number \(B \) violation.
  \item C- and CP-symmetry violation.
  \item Deviation from thermal equilibrium.
\end{enumerate}
Baryon number violation is necessary to produce an excess amount of baryons in an interaction.
C-symmetry violation is necessary so that interactions that produce and excess amount of baryons will not be balanced by interactions that produce an excess amount of anti-baryons.
CPV is necessary otherwise equal numbers of left-handed baryons and right-handed anti-baryons would be produced, as well as equal numbers of left-handed anti-baryons and right-handed baryons.
Deviation from thermal equilibrium is necessary otherwise CPT symmetry would assure compensation between processes increasing and decreasing the baryon number~\cite{ShaposhnikovFarrar1993BAU}.
The SM does not have an adequate amount of CP violation to account for the observed asymmetry, nor does it have out-of-equilibrium processes.
However, with two or more Higgs doublets, EWBG is achieveable~\cite{Bochkarev1990EWBG2HDM}, with the added bonus that the proposed sub-TeV dynamics can be tested at the LHC.
This motivates us to consider models that are SM extensions featuring two Higgs doublets, also known as two Higgs doublet models (2HDMs).