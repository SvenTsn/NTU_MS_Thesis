\chapter{Introduction}
\label{ch:intro}

One of the biggest unanswered questions of particle physics is that of baryogenesis.
Specifically, if electroweak baryogenesis (EWBG)~\cite{EWBG} were to occur, one would require very large CP violation (CPV) beyond the Standard Model (BSM), since the SM currently houses all its CPV in the CKM matrix~\cite{PDG}.
However, such large BSM-CPV should have led to new discoveries at the LHC, which evidently is \textit{not} what has been observed.
Moreover, in the low-energy precision frontier, electric diploe moments (EDMs) provide a \textit{litmus test} for CPV effects, and experiments have achieved higher and higher precision without discoveries, setting ever more stringent bounds.

\section{The Standard Model of Particle Physics}
The current working theory in the realm of particle physics is the Standard Model (SM).

\section{Limits of the Standard Model}
The SM is a powerful theory, providing a unifying framework for three of the four fundamental forces, and producing many verified predictions.
To quote from the textbook \textit{Modern Particle Physics} by Mark Thomson~\cite{thomsonParticleTextbook},
``[The Standard Model] is a model constructed from a number of beautiful and profound theoretical ideas
put together in a somewhat \textit{ad hoc} fashion in order to reproduce the experimental data.''
However, amidst all its strength, there are still several open questions that the SM has not been able to answer.
