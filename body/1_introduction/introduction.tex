\chapter{Introduction}
\label{ch:intro}

One of the biggest unanswered questions of particle physics is that of baryogenesis.
Specifically, if electroweak baryogenesis (EWBG)~\cite{EWBG} were to occur, one would require very large CP violation (CPV) beyond the Standard Model (BSM), since the SM currently houses all its CPV in the CKM matrix~\cite{PDG2020}.
However, such large BSM-CPV should have led to new discoveries at the LHC, which evidently is \textit{not} what has been observed.
Moreover, in the low-energy precision frontier, electric diploe moments (EDMs) provide a \textit{litmus test} for CPV effects, and experiments have achieved higher and higher precision without discoveries, setting ever more stringent bounds.

\section{The Standard Model of Particle Physics}
The current working theory in the realm of particle physics is the Standard Model (SM).

\section{Limits of the Standard Model}
The SM is a powerful theory, providing a unifying framework for three of the four fundamental forces, and producing many verified predictions.
However, it is clear that the SM is \textit{definitely not} the ultimate framework, and there is definitely room for extensions and modifications.
To quote from the textbook \textit{Modern Particle Physics} by Mark Thomson~\cite{Thomson2013ParticleTextbook},
``[The Standard Model] is a model constructed from a number of beautiful and profound theoretical ideas
put together in a somewhat \textit{ad hoc} fashion in order to reproduce the experimental data.''
Amidst all its strength, there are still several open questions that the SM has not been able to answer.
In particular, the baryon asymmetry of the universe (BAU) problem is about the observed imbalance in baryonic matter and anti-baryonic matter amounts.
Current cosmological theory predict that the Big Bang started with an almost equal number of quarks and antiquarks~\cite{Sarkar2007AstroparticlePhysics}.
It is further hypothesized that during the expansion and cooling of the universe, some physical process occured that created the observed BAU.
This process is known as baryogenesis~\cite{Liddle2015Cosmology} (formerly baryosynthesis~\cite{BarrowTurner1981Baryosynthesis,Turner1981GUT}).
In particular, electroweak baryogenesis (EWBG) is the theory where such a baryogenesis process takes place during the electroweak phase transition (EWPT).
Inspired by the discoveries of the Cosmic Microwave Background (CMB)~\cite{PenziasWilson1965CMB} and CPV in the neutral Kaon system~\cite{CroninFitch1964KaonCPV},
Andrei Sakharov proposed~\cite{Sakharov1967BAU} a set of three necessary conditions for BAU, and consequentially baryogenesis, to occur:
\begin{enumerate}
  \item Baryon number \(B \) violation.
  \item C- and CP-symmetry violation.
  \item Deviation from thermal equilibrium.
\end{enumerate}
Baryon number violation is necessary to produce an excess amount of baryons in an interaction.
C-symmetry violation is necessary so that interactions that produce and excess amount of baryons will not be balanced by interactions that produce an excess amount of anti-baryons.
CPV is necessary otherwise equal numbers of left-handed baryons and right-handed anti-baryons would be produced, as well as equal numbers of left-handed anti-baryons and right-handed baryons.
Deviation from thermal equilibrium is necessary otherwise CPT symmetry would assure compensation between processes increasing and decreasing the baryon number~\cite{ShaposhnikovFarrar1993BAU}.
The SM does not have an adequate amount of CP violation to account for the observed asymmetry, nor does it have out-of-equilibrium processes.
However, with two or more Higgs doublets, EWBG is achieveable~\cite{Kuzmin1985EWBG}, with the added bonus that the proposed sub-TeV dynamics can be tested at the LHC.
This motivates us to consider models that are SM extensions featuring two Higgs doublets, also known as two Higgs doublet models (2HDMs).