\chapter{Introduction}
\label{ch:intro}

The Standard Model of particle physics has solidified itself over the past 50-or-so years as the fundamental theory of everything quantum.
This is both a blessing and a curse for particle physicists. 
While the SM provides a very powerful framework to explain a plethora of phenomena, 
it is so accurate in explaining what it \emph{can} explain 
yet provides next to no direction in how to probe the questions it \emph{cannot} explain.
Beyond the Standard Model (BSM) theories are a dime a dozen nowadays, 
all trying their best to answer these unanswered questions.
A question that is of particular importance is that of the baryon asymmetry of the universe (BAU),
and namely its driving process baryogenesis,
which is how the matter-antimatter imbalance came to be.
This opens the door to large BSM charge-parity violation (CPV),
since the SM CPV, which is completely housed in the CKM matrix~\cite{PDG2022},
is not enough to explain what we have observed.
However, large BSM-CPV should have led to new discoveries at the LHC, which evidently is \textit{not} what has been observed.
In this thesis, we explore CPV through a General Two Higgs doublet model ({\gthdm}), 
which is an extension of the SM by introducing an extra Higgs doublet.
We first elucidate the theortical potential of the model in generating BAU,
then turn our focus to precision experiments of the EDMs of fundamental particles.
These EDMs are very sensitive to CPV effects, so they can serve as a \emph{litmus test} of the model.
We introduce a method of satisfying the precision bounds while still maintaining adequate efficiency for BAU in {\gthdm}.
In the end, we comment on the experimental prospects for these EDM experiments,
as well as the prospects for {\gthdm} under the new and improved experiments in the coming decade or two.

\section{The Standard Model of Particle Physics}
The current working theory in the realm of particle physics is the Standard Model (SM). 
The SM is a quantum field theory (QFT) that provides a unified framework for particles and their interactions.
Particles are described as localized excitations of their coorresponding quantum fields,
while their interactions (``forces'') manifest from local gauge symmetries of said fields.
In this section we first give a brief historical rundown of how the SM developed into the form it has today,
then we provide a top-down description of the important parts of the mathematical and physical framework.

\subsection{Brief History}
The late 1920s up till the early 1980s was a booming period for modern physics.
Multiple theoretical and experimental frontiers were advancing in their respective ways,
while simultaneously influencing each other,
eventually culminating in a unified theory with significant experimental verification.

[1925] Born, Heisenberg, Jordan~\cite{BornJordan1925Quantization, BornHeisenbergJordan1926Quantization} canonical quantization of the electromagnetic field (inspired by de Broglie~\cite{DeBroglie1925Quantization}).
[1927] Dirac "The Quantum Theory of the Emission and Absorption of Radiation"~\cite{Dirac1927QED}.
[1928] Dirac equation~\cite{Dirac1928DiracEquation}, (implied existence of antimatter).
[1932] Positron discovered by Anderson~\cite{Anderson1933Positron}.
[1934] Fermi theory of beta decay~\cite{Fermi1934BetaDecay}.
[1936] Muon discovered by Anderson and Neddermeyer~\cite{AndersonNeddermeyer1936Muon} (confirmed by Street and Stevenson~\cite{StreetStevenson1937Muon} in 1937).
[1940s-50s] Formalization of QED as U(1) gauge theory, renormalization~\cite{Tomonaga1946QED, Schwinger1948QED-1, Schwinger1948QED-2, Feynmann1949QED-1, Feynmann1949QED-2, Feynmann1950QED-3, Dyson1949QED-1, Dyson1949QED-2}.
[1946] Identity of the muon as a heavier partner of the electron elucidated~\cite{ConversiPanciniPiccioni1947Muon} (``start of modern particle physics'' c.f. Alvarez\footnote{1968 Nobel Prize lecture}).
[1954] Yang and Mills non-Abelian gauge theory~\cite{YanMills1954NonAbelianGaugeTheory}.
[1957] Lee, Yang~\cite{LeeYang1956ParityViolation}, Wu~\cite{WuEtAl1957ParityViolation} P-violation in the weak interaction.
[1961] Glashow combined the electromagnetic and weak interactions~\cite{Glashow1961Electroweak}.
[1964] PRL symmetry breaking papers~\cite{BroutEnglert1964HiggsMechanism, Higgs1964HiggsMechanism, GHK1964HiggsMechanism} established the Higgs mechanism\footnote{
  If one were to properly credit all the scientists, it should be called the 
  Brout-Englert-Higgs-Guralnik-Hagen-Kibble mechanism, 
  or perhaps the BEHGHK mechanism for short.} 
for mass inspired by superconductivity (\cite{BCS1957Superconductivity} for the BCS theory of superconductivity, \cite{Anderson1963SSB, KleinLee1964SSB} for applying said framework to particle gauge theories).
[1964] Gell-Mann~\cite{GellMann1964QuarkModel} and Zweig~\cite{Zweig1964QuarkModel} quark model.
[1965] Tomonaga, Schwinger, Feynmann Nobel Prize for QED.
[1967] Weinberg~\cite{Weinberg1967Electroweak} and Salam~\cite{Salam1968Electroweak} incorperated the Higgs mechanism into Glashow's electroweak theory.
[1968] ``Partons'' discovered at SLAC~\cite{SLAC1969Partons-1, SLAC1969Partons-2}.
[1970] Glashow, Iliopoulos, Maiani GIM mechanism~\cite{GIM1970GIMMechanism} predicting the charm quark (charm quark predicted earlier in 19 by Bjorken and Glashow~\cite{BjorkenGlashow1964CharmPrediction}).
[1973] Z boson interaction discovered at CERN~\cite{CERN1973ZInteraction-1, CERN1973ZInteraction-2, CERN1973ZInteraction-3}.
[1973] Gross and Wilczek~\cite{GrossWilczek1973AsymptoticFreedom} and Politzer~\cite{Politzer1973AsymptoticFreedom} asymptotic freedom for non-Abelian gauge theories,
which solidified QCD as the theory of strong interactions in the few years that followed.
[1973] Kobayashi and Masakawa~\cite{KobayashiMasakawa1973CKMMatrix} CP violation from Cabibbo~\cite{Cabibbo1963CabbiboMatrix}.
[1974] Charm quark discovered at SLAC~\cite{SLACLBL1974CharmDiscovery}, BNL~\cite{BNLMIT1974CharmDiscovery}, and Friscati (ADONE)~\cite{Frascati1974CharmDiscovery}.
1975-76 Tau discovered at SLAC~\cite{PerlSLAC1975TauEvidence-1, PerlSLAC1976TauEvidence-2, PerlSLAC1977TauDiscovery}.
[1977] Bottom quark discovered at Fermilab~\cite{Fermilab1977BottomQuarkDiscovery}.
[1979] Glashow, Salam, Weinberg Nobel Prize for complete electroweak theory.
[1983] Physical W and Z bosons discovered~\cite{CERN1983WDiscovery-1, CERN1983WDiscovery-2, CERN1983ZDiscovery-1, CERN1983ZDiscovery-2}.
[1995] Top quark discovered at Fermilab~\cite{Fermilab1995TopQuarkDiscovery-1,Fermilab1995TopQuarkDiscovery-2}.
[2012] Higgs boson discovered at CERN~\cite{CMS2012HiggsDiscovery, ATLAS2012HiggsDiscovery}.
[2013] Higgs and Englert Nobel Prize for Higgs mechanism.
[2013-now] No significant experimental discovery. Theorists running around like headless chickens.

As an interesting anecdote, the name ``Standard Model'' was coined by Pais and Treiman in their 1975 paper~\cite{PaisTreiman1975SMName}
for the unified electroweak theory with four quarks,
but Weinberg claims~\cite{WeinbergInterview2018} that he first used the term in 1973 during a talk in Aix-en-Provence in France.

\subsection{Theoretical Overview}
Lagrangian formalism, QFT.
Feynmann diagrams.
Poincar\'{e}, SU(3), SU(2), U(1) symmetries.
Free field + interaction terms.
Strong interaction.
Electroweak interaction.
Higgs mechanism and SSB.
Particle content (table?).

\section{Limits of the Standard Model}
The SM is a powerful theory, providing a unifying framework for three of the four fundamental forces, and producing many verified predictions.
However, it is clear that the SM is \textit{definitely not} the ultimate framework, and there is definitely room for extensions and modifications.
To quote from the textbook \textit{Modern Particle Physics} by Mark Thomson~\cite{Thomson2013ParticleTextbook},
``[The Standard Model] is a model constructed from a number of beautiful and profound theoretical ideas
put together in a somewhat \textit{ad hoc} fashion in order to reproduce the experimental data.''
Amidst all its strength, there are still several open questions that the SM has not been able to answer.
In particular, the baryon asymmetry of the universe (BAU) problem is about the observed imbalance in baryonic matter and anti-baryonic matter amounts.
Current cosmological theory predict that the Big Bang started with an almost equal number of quarks and antiquarks~\cite{Sarkar2008AstroparticlePhysics}.
It is further hypothesized that during the expansion and cooling of the universe, some physical process occured that created the observed BAU.
This process is known as baryogenesis~\cite{Liddle2015Cosmology} (formerly baryosynthesis~\cite{BarrowTurner1981Baryosynthesis,Turner1981GUT}).
In particular, electroweak baryogenesis (EWBG) is the theory where such a baryogenesis process takes place during the electroweak phase transition (EWPT)~\cite{Kuzmin1985EWBG, Cohen1990EWBG}.
Inspired by the discoveries of the Cosmic Microwave Background (CMB)~\cite{PenziasWilson1965CMB} and CPV in the neutral Kaon system~\cite{CroninFitch1964KaonCPV},
Andrei Sakharov proposed~\cite{Sakharov1967BAU} a set of three necessary conditions for BAU, and consequentially baryogenesis, to occur:
\begin{enumerate}
  \item Baryon number \(B \) violation.
  \item C- and CP-symmetry violation.
  \item Deviation from thermal equilibrium.
\end{enumerate}
Baryon number violation is necessary to produce an excess amount of baryons in an interaction.
C-symmetry violation is necessary so that interactions that produce and excess amount of baryons will not be balanced by interactions that produce an excess amount of anti-baryons.
CPV is necessary otherwise equal numbers of left-handed baryons and right-handed anti-baryons would be produced, as well as equal numbers of left-handed anti-baryons and right-handed baryons.
Deviation from thermal equilibrium is necessary otherwise CPT symmetry would assure compensation between processes increasing and decreasing the baryon number~\cite{ShaposhnikovFarrar1993BAU}.
The SM does not have an adequate amount of CP violation to account for the observed asymmetry, nor does it have out-of-equilibrium processes.
However, with two or more Higgs doublets, EWBG is achieveable~\cite{Bochkarev1990EWBG2HDM}, with the added bonus that the proposed sub-TeV dynamics can be tested at the LHC.
This motivates us to consider models that are SM extensions featuring two Higgs doublets, also known as two Higgs doublet models (2HDMs).