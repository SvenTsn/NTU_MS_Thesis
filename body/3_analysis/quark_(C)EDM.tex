\chapter{Electric and Chromo-electric Dipole Moment of Quarks}
\label{ch:quark(C)EDM}

After the analysis for leptons, we turn our gaze towards EDMs involving quarks. 
The prime candidate in this case would be the neutron EDM (nEDM).
As mentioned in the theory section, there are additional chromo-EDM and Weinberg term contributions to take into account.
We use the recent formula~\cite{Hisano15}
\begin{equation}
  d_n = - 0.20\,d_u + 0.78\,d_d + e\,(0.29\,\tilde d_u + 0.59 \tilde d_d) + e\,23\;{\rm MeV}\,C_W
\end{equation}
to estimate the nEDM.
We evaluate the contributions to \(\tilde{d}_{u, d} \) and \(C_{W} \) in {\gthdm} by following Refs.~\cite{Abe14} and \cite{JungPich14}, with discussion on theoretical uncertainties found in Ref.~\cite{KanetaEtAl23}.
We present the results for nEDM in Fig.~\ref{fig:nEDM-fixed},
as well as combined results for eEDM and nEDM in the range \(r \in [0.6, 0.8] \) in Fig.~\ref{fig:nEDM-eEDM},
with the ansatz applied.
Interestingly, our predictions for nEDM are not too far below the current experimental bound.
We relax the ansatz for \(\rho_{uu} \), and explore the range of \(\order{\lambda_{u}} \). 
We present our results.
We comment on the \textit{natural} cancellation present between \(\rho_{uu} \) and \(\rho_{tt} \).
When viewed together with eEDM, the experimental prospects for discovery are promising.