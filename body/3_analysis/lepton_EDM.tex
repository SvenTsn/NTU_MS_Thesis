\chapter{Electric Dipole Moment of Leptons}
\label{ch:leptonEDM}

Naturally, following the framework laid out previously, we want to perform calculations on the various leptons.
A brief review of past experimental results shows that the experimental development of electron EDM (eEDM) over the past few years has been remarkably rapid.
Just earlier last year, JILA~\cite{JILA23} has surpassed the previous bound from ACME~\cite{ACME18} and pushed the precision of eEDM down to \(|d_{e}| < 4.1 \times 10^{-30}\) \(e\,\mathrm{cm} \).
It is noteworthy to point out that these eEDM experiments are relatively small in scale, ``tabletop experiments'' even when compared to behemoths like the LHC, which makes the extreme precision achieved all the more impressive.
For the electron, an extensive study in eEDM in {\gthdm} can be found in the 2018 and 2020 papers of Fuyuto, Senaha, and Hou (Refs.~\cite{FSH20}).
Our investigation on eEDM is essentially an extension of the 2020 paper to a larger parameter space.
Motivation due to tension between baryogenesis v.s. precision measurements.
Cancellation mechanism, re-empasize ``rule of thumb''.
Present results.
Comment on prospect of future experiments, as well as interplay with neutron EDM, which will be discussed later.

After the electron, we move on to its slightly heavier cousin, the muon. 

Lastly, we analyze the heaviest lepton, the tau. 
On the experiment front, the precision of tauEDM measurements are still pretty low.
As seen in \figref{fig:tauEDM}, when performing the same analysis as the muon, our predicted values are still several orders of magnitude below current experimental results.
Further precision or methodology improvements are required for a more fruitful analysis of tauEDM, so we just present our results here without much further comment.
